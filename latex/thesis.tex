% Document template suitable for use as a LaTeX master-file for master's
% thesis in University of Turku Department of Computing.

% HOW TO USE? See https://ttweb.utugit.fi/thesis/doc/overview/

\documentclass[language=english,version=draft,mainfont=none,minted=true]{utuftthesis}
\setcounter{secnumdepth}{2}
\setcounter{tocdepth}{2}
\usepackage{float}
\usepackage[caption=false]{subfig}
\usepackage{etoolbox}
\usepackage{setspace}


% Define the algorithm environment
%\makeatletter
\providecommand\textquotedblplain{%
  \bgroup\addfontfeatures{Mapping=}\char34\egroup}
\providecommand{\tabularnewline}{\\}
\floatstyle{ruled}
\newfloat{algorithm}{htbp}{loa}
\providecommand{\algorithmname}{Algoritmi}
\floatname{algorithm}{\protect\algorithmname}
%\makeatother

% \renewcommand\floatpagefraction{.85}
% \renewcommand\topfraction{.1}

% Custom commands
\newcommand{\todo}[1]{\textcolor{red}{TODO: #1}}
\newcommand{\rly}[1]{\textcolor{red}{CHECK:} \textcolor{gray}{#1}}
\newcommand{\remove}[1]{\textcolor{blue}{Remove?} \textcolor{gray}{#1}}
\newcommand{\refsource}[1]{Listing \ref{#1}}
\newcommand{\acronym}[2]{\nomenclature{#1}{#2} #2 (#1)}

\addbibresource{Bibliography.bib}

\begin{document}

\pubyear{2022}
\pubmonth{5}
\publab{Software Engineering}
\pubtype{di}
% \title{Managing side effects with functional effect systems}
\title{Purely functional statically typed programming with side effects}
\author{Jaakko Paju}
\supervisors{Jaakko Järvi}

\maketitle

\keywords{functional programming, side effect, algebraic effect, monad, Scala, ZIO}

\begin{abstract}
\new{
The management of side effects is a crucial aspect of modern programming, especially in concurrent and distributed systems. This thesis presents different approaches to managing side effects in programming languages, specifically focusing on unrestricted side effects, monads, and algebraic effects and handlers.
Unrestricted side effects, used in mainstream imperative programming languages, can make programs difficult to reason about. Monads offer a solution to this problem by describing side effects in a composable and referentially transparent way. Algebraic effects and handlers are able to address some of the shortcomings of monads by providing a way to model effects in more modular and flexible way.
The thesis focuses on ZIO, a Scala library for concurrent and asynchronous programming, which revolves around a ZIO monad with three type parameters. With those three parameters ZIO is able to encode majority of the practically useful effects in a single monad. ZIO takes inspiration from algebraic effects and combines those with monadic effects. The library provides a range of features such as concurrency primitives, error handling, and resource management.
The thesis discusses the advantages and disadvantages of each approach and compares them based on factors such as expressiveness, safety, and constraints it places. Additionally, the thesis presents examples of using ZIO to manage side effects in practical scenarios, highlighting its strengths over other approaches.
}

\todo{Case study?}

\end{abstract}


\tableofcontents % mandatory

% \listoffigures % if you want a list of figures
% \listoftables % if you want a list of tables
\listofacronyms % if you want a list of acronyms

% change the name if the default doesn't sound right
\renewcommand{\algorithmname}{\listingscaption}

\newcommand{\todo}[1]{\textcolor{red}{TODO: #1}}
\newcommand{\rly}[1]{\textcolor{red}{CHECK:} \textcolor{gray}{#1}}

\newcommand{\refsource}[1]{Listing \ref{#1}}

\newcommand{\acronym}[2]{\nomenclature{#1}{#2} #2 (#1)}

% Minted 
\AtBeginEnvironment{algorithm}{\singlespacing}
\setmintedinline{breaklines}
\newcommand{\inlinecode}[1]{\mintinline[breaklines=false]{text}{#1}}
\newcommand{\inlinescala}[1]{\mintinline[breaklines=false]{scala}{#1}}
\newcommand{\inlinehaskell}[1]{\mintinline[breaklines=false]{haskell}{#1}}

% Citing
\newcommand{\titlecite}[1]{\citetitle{#1}~\cite{#1}}
% \newcommand{\titlecite}[1]{\citetitle{#1}}

% Content
\chapter{Introduction} \label{Introduction}

Modern programs interact with their environment, such as users, files, databases, message buses, and/or other applications. Programs should be able to serve hundreds, thousands, or sometimes even millions users at the same time, utilizing the underlying hardware efficiently. The programs are expected to be available and working every day of the year, around the clock. These programs are expected to be robust and resilient, meaning they should react to failures in a predictable and well-defined manner. At the same time, programs should be fast to develop and modify when adding new features or changing existing ones, i.e. applications are desired to be modular.

This is no easy task for a programmer to undertake. Many of the described problems are related to managing both side effects and concurrency, and exceptions arising from those. Side effects, also known as computational effects or just effects, are a byproduct of calling a function that causes or observes changes in its environment. Concurrency is the ability to interleave several units of work to be executed at the same time.

Modular and expressive management of side effects, errors and concurrency is something that current, imperative, mainstream languages do not excel at. In the academia, however, are several techniques that make it possible to work with effects in a compositional and expressive way. More sophisticated methods for managing effects are based on functional programming. Even though the theoretical foundations of functional programming date back to almost a hundred years, when lambda calculus was invented in the 1930s, functional programming languages have not became mainstream. All of today's most widely used programming languages, as ranked by TIOBE Index~\cite{tiobe-index}, are fundamentally imperative.

Many functional concepts, however, have been recognized to be valuable in modern software development. Functional programming promises case of reasoning about program behavior, immutability gives referential transparency and equational reasoning, and composability. These functional concepts are well-suited for handling effects, concurrency and modeling complex business logic, which are the core of many modern applications. Features like immutability, lambdas and higher-order functions, have found their way to imperative and object-oriented mainstream languages like JavaScript, Python, Java, and C\#.

\new{
Functional features are currently disrupting the field of programming. The purpose of this thesis is to analyze and understand how these features can be utilized. The aim is to bridge the cap between solutions present in academia and the technologies used in the industry by studying different methods of managing effects from a practical perspective. The thesis studies three different approaches to side effects; unrestricted side effects, monads, and algebraic effects and handlers. In addition, a Scala library called ZIO, which applies the approaches in question, is studied. The methods are studied in terms of how they affect the implementation of programs.
}
Research questions are:
\begin{description}
    \item[RQ1:] How expressive and compositional the method is?
    \item[RQ2:] What are the safety guarantees the method offers?
    \item[RQ3:] Does the method place constraints on how programs can be written?
\end{description}

\new{
Different ways of managing side effects can also be researched by studying how it affects testing, or approaching the issue from a social perspective. For example, monads and algebraic effects have properties that may facilitate testing. On the other hand, the adoption of a method for managing side effects is probably influenced by how familiar or unfamiliar the developers perceive it to be. However, testing and social aspects were left out of the scope of this thesis.
}

\todo{Korjaa tähän muuttunut rakenne}
Chapter 2 studies the definition of effects and introduces several common types of effects. Concurrency, related problems, and how it is implemented is discussed. Next the chapter introduces how effects are included in programming languages, and how they can be managed. Also Scala and its relevant features are introduced in this chapter. Chapter 3 introduces ZIO and explores how it approaches effect management. \todo{Lisää maininta case study -chapterista?} The last chapter compares the properties of different methods for managing side effects and draws conclusions from them.

\todo{Kontribuutiot?}



\section{Scala} \label{scala}
Scala is a high level, statically typed, compiled, and garbage collected programming language, that is both functional and object-oriented. It is eagerly evaluated by default, but supports also lazy evaluation. The first release was in 2004 and the latest version is 3, which was released in 2021. Version 3 is exclusively used in this thesis. The initial and current lead designer of the language is Martin Odersky, a professor at the École polytechnique fédérale de Lausanne. Scala's roots are thus in an academia, but its approach is pragmatic.

Scala is most commonly run on the \acronym{JVM}{Java Virtual Machine}, but also JavaScript and native code are supported compilation targets. When running on the JVM it is possible to use Java code directly from Scala. The Scala standard library even contains functions for converting Java data types to their Scala counterparts. This gives access to huge number of Java libraries.

Scala aims to blend the \acronym{FP}{Functional programming} and \acronym{OOP}{Object-oriented programming} paradigms and as a result has features from both. Many OOP concepts like classes, objects, interfaces and subtype polymorphism are supported. In fact, every value in Scala is an object. Scala uses class-based objects with attributes and methods, and supports multiple inheritance. Scala supports generics with lower and upper subtype constraints as well as declaration-site variance. The language also includes many imperative constructs, like loops and mutable variables, that are commonly used in other OO-languages. Perhaps less common in OO-languages, in Scala everything is an expression, including control structures like \inlinecode{if/else}, \inlinecode{try/catch}, and loops. \refsource{scala:basics} demonstrates the basic syntax of Scala.

\begin{algorithm}

\begin{minted}{scala}
val numberZio  = ZIO.succeed(2)
val doubledZio = numberZio.map(n => n * 2)

val sum: UIO[Int] =
  for
    a <- numberZio
    b <- doubledZio
  yield a + b
\end{minted}

\caption{ZIO Basics \label{zio:basics}}
\end{algorithm}

Due to Scala's object-oriented nature, every object is part of a type hierarchy. On top of the hierarchy is \inlinecode{Any}, which is the supertype of all other types. Below \inlinecode{Any} is \inlinecode{Matchable}, which marks values suitable for pattern matching. \inlinecode{Matchable} has two subtypes: \inlinecode{AnyVal}, a supertype for all value types, and \inlinecode{AnyRef}, a supertype for all reference types. \inlinecode{Null} is a subtype of all reference types, except when \emph{explicit nulls} -feature is enabled and \inlinecode{Null} becomes a subtype of \inlinecode{Any}. Scala also has a bottom type \inlinecode{Nothing} that is a subtype of every type. No values of type \inlinecode{Nothing} can ever exist at runtime so the type reflects the absence of a value, for example in the case of infinite recursion or loop, or when the expression throws an exception. The diagram in Figure \ref{fig:scala-type-hierarchy} depicts the type hierarchy.

\begin{figure}
    \centering
    \includegraphics[width=\textwidth]{images/type-hierarchy}
    \caption{Scala 3 type hierarchy.}
    \label{fig:scala-type-hierarchy}
\end{figure}

Variance defines the rules on how the subtype relationships between parameterized types are dependent on the subtype relationship on the type on which it is parameterized. Variance has three variants: \emph{invariance}, \emph{covariance}, and \emph{contravariance}. Invariance means that subtyping relationships present in type parameters are not applied to the parameterized type at all. Covariance states that the subtype relationship of the parameterized type are in the same direction as a type parameter's subtype relationship. Contravariance means that the subtype relationship between parameterized types are the opposite way compared to the subtype relationships of the type parameter. When \inlinecode{Sub} is a subtype of \inlinecode{Super} and \inlinescala{F[_]} is any parameterized type, then
\begin{itemize}
    \item Under covariance, \inlinescala{F[Sub]} is a subtype of \inlinescala{F[Super]};
    \item Under contravariance, \inlinescala{F[Super]} is a subtype of \inlinescala{F[Sub]}; and
    \item Under invariance, \inlinescala{F[Sub]} and \inlinescala{F[Super]} have no subtyping relationship.
\end{itemize}

Covariance is applicable in parameterized types that contain, store, or produce values, in other words the type parameter is in covariant position. Contravariance is applicable in the opposite situation, where values of the type parameter are consumed, i.e. the type parameter appears in a function parameter list, and is said to be in contravariant position. Invariance is useful in situations where it does not make sense for the parameterized type to have inheritance based on type parameter, or when the parameterized type is a mutable, or the type parameter appears both in covariant and contravariant positions. A parametric type with multiple type parameters could declare each type parameter with different variance, for example functions in Scala are contravariant in their input type(s) and covariant in their result type.

An infamous example of a mutable covariant type is the primitive array type in Java and C\#.
These arrays must perform a runtime type check when adding elements to the array, and throw an exception if the type of the element is not compatible with the array, as demonstrated in \refsource{mutable-covariance}. To avoid these mandatory costs and checks, mutable collections in Scala are invariant. Immutable collections and containers, such as \inlinecode{Option} or \inlinecode{Either} are covariant in Scala.

\begin{algorithm}

\begin{minted}{java}
String[] a = new String[1];	
Object[] b = a; // String is a subtype of Object, so this is legal
b[0] = 1; // Runtime exception since cannot add Integer to String[]
\end{minted}

\caption{Covariance in mutable types, like Java primitive array, is problematic \label{mutable-covariance}}
\end{algorithm}

Programming languages differ in the way variance annotations are defined and used. Variance annotations in C\# and Scala are with the parameterized type. On the other hand, in Java one defines variance only when using a parameterized type. The former is called \emph{declaration-site} variance, demonstrated in \refsource{declaration-site-variance} and the latter is called \emph{use-site} variance, demonstrated in \refsource{use-site-variance}. Approaches deviating from these exist, for example TypeScript tries to infer variance, but it has optional declaration-site annotations from version 4.7 onwards. Kotlin has declaration-site variance by default but it emulates some parts of use-site variance with type projections.

\begin{algorithm}

\begin{minted}{scala}
class Invariant[A]      // Invariance is the default
class Covariant[+A]     // Covariance denoted with +
class Contravariant[-A] // Contravariance denoted with -
\end{minted}

\caption{Scala uses declaration-site variance, where the variance of a parameterized type is denoted in its type definition  \label{declaration-site-variance}}
\end{algorithm}

\begin{algorithm}

\begin{minted}{java}
interface Supertype {}
interface Subtype extends Supertype {}

void invariant(List<Supertype> list) {
    /* Get and set list values */
}
void covariant(List<? extends Supertype> list) {
    /* Only get list values */
}
void contravariant(List<? super Subtype> list) {
    /* Only set list values */
}
\end{minted}

\caption{Java has use-site variance, where the desired variance is declared when using the parameterized type. \label{use-site-variance}}
\end{algorithm}

Invariance is the default in Scala and it does not require an explicit annotation. Covariance is declared with a \inlinecode{+} sign before each type parameter. Since contravariance could be seen as the opposite of covariance, it is denoted with a \inlinecode{-} sign.

Many features and principles from functional programming are not only available, but also encouraged in Scala. Pattern matching, first-class functions (\refsource{scala:lambdas}), and tail recursion are all supported and heavily utilized in idiomatic Scala programs. Immutable variables, collections and data-structures are the default way of writing Scala, even though mutable counterparts are also available. Functional data modeling is achieved with the use algebraic data types built into the language. Even though Scala embraces functional programming and imperative code is generally discouraged, introducing arbitrary side effects is possible.

\begin{algorithm}
\begin{minted}{scala}
val ns = List(1, 2, 3)

val mapped1 = ns.map(n => n + 1)
val mapped2 = ns.map(_ + 1) // Same as above in shorter form

val sum1 = ns.foldLeft(0)((x, y) => x + y)
val sum2 = ns.foldLeft(0)(_ + _) // Same as above in shorter form
\end{minted}

\caption{Long and short form of anonymous functions in Scala \label{scala:lambdas}}
\end{algorithm}

In addition to ordinary functions, Scala has a specific function type called \\\inlinecode{PartialFunction} for representing functions that are not defined for all values of their input types. It is a subtype of the (normal) function type, to which it adds the method \inlinecode{isDefinedAt}, which must be used to check before every function call whether the function is defined for the given value. \refsource{scala:partial-function} shows how to define and use partial functions.

\begin{algorithm}
\begin{minted}{scala}
val someEvensMultipliedByTen: PartialFunction[Option[Int], Int] = {
  case Some(n) if n % 2 == 0 => n * 10
}

val opts  = List(None, Some(2), None, Some(3), Some(4))
val somes = opts.collect(someEvensMultiplied) // List(20, 40)
\end{minted}

\caption{Partial functions in Scala \label{scala:partial-function}}
\end{algorithm}

Some functional languages, such as Haskell, have a special syntax for monadic computations. Scala also provides this syntactic sugar in a form of \inlinecode{for} comprehensions, demonstrated in \refsource{monad:for-syntax}. For comprehension is compatible with any data type that has \inlinecode{map} and \inlinecode{flatMap} methods defnied, such as \inlinecode{Option}, \inlinecode{Either}, and \inlinecode{ZIO} (Chapter \ref{zio}). These required methods can be added to any type by using extension methods.

Extension methods, which allow adding methods to a class separately from its definition, are one of Scala's more advanced features. Other state-of-the-art features of Scala include operator overloading and infix operator- and method syntax, higher kinded and dependent types, type lambdas, as well as powerful meta programming capabilities. Scala 3 introduced more cutting edge features such as automatic type class derivation and union and intersections types.

Probably the most distinguishing feature in Scala is its system of implicits and other contextual abstractions arising from that. A function can mark some of its parameters as implicit and the compiler will try to figure out that parameter from the enclosing scope by its type without programmer explicitly passing an argument for that parameter. Originally implicit parameters were introduced to achieve similar behavior as Haskell's type classes. Implicits can, however, also be used for other purposes, such as implicit conversions, context propagation, extension methods, and proving subtyping relationships between generic type parameters at compile-time.~\cite{tc-as-objects}

Syntactically to use implicits, a function can mark some of its parameters as implicit with the keyword \inlinecode{using}. When the function is called, the compiler tries to find a value marked as implicit, with the keyword \inlinecode{given}, from the enclosing scope. If all requested values are found, they are automatically applied as arguments. If any of the implicit parameters is not found, compilation error is reported. \refsource{scala:implicits} shows the function \inlinecode{summon} that searches for an implicit value by type, demonstrating the definition and use of implicit parameters.

\begin{algorithm}
\begin{minted}{scala}
case class Person(age: Int, name: String)

// Define type class
trait Show[A]:
  extension (a: A) def show: String

// Define type class instance
given Show[Person] with
  extension (a: Person)
    def show: String = s"${a.name} is ${a.age} years old"

// Use the type class
def showAll[A: Show](as: List[A]): List[String] =
  as.map(a => a.show)
\end{minted}

\caption{Implicits could be used to encode type classes \label{scala:typeclasses}}
\end{algorithm}

Another advanced feature utilizing implicit resolution is the ability of the Scala compiler to prove type equality or subtype relationships. The class \inlinescala{=:=[From, To]} is for type equality and \inlinescala{<:<[From, To]} for subtype relationship. Both classes extend a function \inlinescala{From => To}, and can be used to transform types. Types with two type parameters could be used as infix in Scala, for example type equality could be written \inlinescala{A =:= B}. When requesting an implicit parameter of either of the types above, Scala compiler synthesizes an instance if the type relationship holds, otherwise reports a compilation error. The act of proving type relationships is said to be \emph{witnessing}, and a common practice is to name the implicit parameter as \emph{evidence}. The feature is useful, for example, when defining functions that make sense only for specific types, as demonstrated in \refsource{scala:witness}, where only nested \inlinecode{Maybe} types could be flattened.

\begin{algorithm}
\begin{minted}{scala}
enum Maybe[+A]:
  case Just(a: A)
  case Nothing

  def flatten[B](using evidence: A <:< Maybe[B]): Maybe[B] =
    this match
      case Just(a) => evidence(a)
      case Nothing => Nothing

Maybe.Just(Maybe.Just(1)).flatten // Compiles
Maybe.Just(1).flatten // Error: Cannot prove that Int <:< Maybe[B]
\end{minted}

\caption{Scala compiler witnesses sub type relationship by providing implicit evidence \label{scala:witness}}
\end{algorithm}

Another thing that sets Scala 2 and 3 apart is the introduction of intersection and union types in Scala 3. Intersection types are denoted with the \inlinecode{&} symbol and union types with \inlinecode{|}. Intersection \inlinescala{A & B} means that the resulting type has properties of both \inlinecode{A} \textbf{and} \inlinecode{B}. Union is the dual of intersection, and the resulting type of \inlinescala{A | B} is either \inlinecode{A} \textbf{or} \inlinecode{B}.

Intersection types are commutative, idempotent, and have \inlinecode{Any} as the identity element. Commutativity means that the order of types included in the intersection does not matter --- Scala considers permutations equal. Idempotency means that type intersectioned with itself is equal to the type itself. \inlinecode{Any} as the identity element means that the intersection of any type \inlinecode{A} with \inlinecode{Any} is equal to \inlinecode{A}, since all types themselves are subtypes of \inlinecode{Any}. Expressed as code, laws of intersection types can be proved with the Scala compiler:
\begin{itemize}
    \item Commutativity: \inlinescala{summon[(A & B) =:= (B & A)]}
    \item Idempotency: \inlinescala{summon[A =:= (A & A)]}
    \item Identity: \inlinescala{summon[A =:= (A & Any)]}
\end{itemize}

Like intersection types, also union types are commutative, idempotent, and obey the identity laws. The identity element is \inlinecode{Nothing}: the union of any type \inlinecode{A} with \inlinecode{Nothing} is equal to \inlinecode{A}, since there are no values of type \inlinecode{Nothing}. Again expressed as code, the laws of union types proved by the Scala compiler are:
\begin{itemize}
    \item Commutativity: \inlinescala{summon[(A | B) =:= (B | A)]}
    \item Idempotency: \inlinescala{summon[A =:= (A | A)]}
    \item Identity: \inlinescala{summon[A =:= (A | Nothing)]}
\end{itemize}

\chapter{Background} \label{Background}

Functional programming uses immutable values and mathematical functions, also known as pure functions, to build programs. Similarly to imperative procedures, pure functions take parameters as input and compute some output. Unlike imperative procedures, however, pure functions are \textbf{only} allowed to transform their inputs to outputs and cannot have any other observable effects. Given the same inputs, a pure function must always evaluate to the same outputs. Abstraction and reuse, similarly to in imperative programs, is achieved by composing functions by passing the output of the previous function to the next function's input. The entire program can be seen as a large function composition of all functions used in the program.

A major difference between imperative and functional programming is in how one can reason about procedure or function compositions. Any expression in functional programming can always be \emph{substituted} with its value without changing the meaning of the program. The same does not apply in imperative programming. There is an implicit temporal coupling between imperative statements, since a statement may depend on the state set by previous statements. Because of this, reordering procedure calls or substituting any procedure call with its return value might change the meaning of the program.~\cite[Chapter~1]{sicp}

A program is considered to be \emph{referentially transparent} if it is possible to substitute an arbitrary expression in the program with its corresponding value without changing the meaning of the program in any way. Referentially transparent programs are easier to understand since they enable \emph{equational reasoning}, also known as \emph{local reasoning}. When composing pure functions, one does not have to understand their implementation, because the only effect the function is allowed to have is to return a result. A developer can only focus on the \emph{function's signature} and its specification, that is, what are the inputs and what is the output. Compilers can also take advantage of referential transparency by safely reordering expressions, evaluating expressions at compile time, memoizing results or by completely skipping the evaluation of expressions that are not required.

Referential transparency is one of the biggest differentiating factors between functional and imperative programming. Abandoning referential transparency has wide-reaching implications. In practice, it makes it much more difficult to refactor and develop programs. Developers are required to be more disciplined and to have wider knowledge of the whole program in order to not unintentionally cause defects. This is particularly evident when programming in the presence of concurrency, where side-effects can lead to race conditions and hard-to-reproduce errors.~\cite[Chapter~3]{sicp}

This chapter introduces first what effects are and discusses certain common effect types in more detail. Then it presents what concurrency is, how it can be achieved and what kind of problems it causes. Structured concurrency, a concept for defining semantics on how concurrent workflows interact, is also introduced. Lastly, the history and features relevant to managing side effects of the Scala programming language are introduced.



\section{Effects} \label{effects}
Constructing programs only by composing pure expressions without any notion of impurity is quite limiting, to say the least. To be useful in practice, programs depend on effects. An expression is said to have an effect, if its sole purpose is not to evaluate to a value or if its evaluation requires interacting with the outside world. For example, printing to the console, accessing the system clock or doing IO are all examples of effects. There is no unambiguous and exact definition of what an effect is, although the concept has been given, somewhat differing, characterizations by many.

\textcite{den-lang-specs} suggest that \textquote{A complete program is thought of as an agent that interacts with the outside world, e.g., a file system, and that affects global resources, e.g., the store [mutable memory]}. They continue by stating that every phrase in a program could be classified to either a value or an effect. A value is a referentially transparent expression, while an effect is an interaction with resources allocated for the program. When an effect is encountered, the control is transferred to a \textquote{central authority}. The central authority manages the use of all resources the program has access to. They continue to describe the interaction between an effect and the central authority:
\begin{displayquote}
An effect is most easily understood as an interaction between a sub-expression
and a central authority that administers the global resources of a program. (..) Given an administrator, an effect can be viewed as a message to the central authority plus enough information to resume the suspended calculation.
\end{displayquote}

\textcite{imperative-fp} as well as \textcite{do-be-do-be-do} see the distinction between expressions and effects as \emph{being} vs. \emph{doing}. This observation is quite interesting since it brings up the concept of computations as values. Certain approaches deliberately differentiate computations from values, while some deliberately unify them. It is later discussed how separation of effects from values applies to monadic effects and algebraic effects with handlers, together with the concept of a central authority presented earlier.

Different effects could be categorized as \emph{internal} or \emph{external}. Unlike internal effects, external effects can be observed from the outside. In the context of a whole program, the only external effect is IO, while other effects are internal. In the context of a function, matters are more complicated since effects such as mutable state, raising exceptions, and concurrency can be both internal or external, depending on the specific situation.


\subsection{Mutability} 
Mutability means that the program is able to change the state of the program, usually by mutating data stored in some memory location, and that it is possible to detect a state change by observing the changed value. Several control structures and language features require mutability. The destructive assignment operation found in almost every mainstream language is by definition mutation.~\cite[Chapter~3]{sicp} Looping constructs such as \inlinecode{for}- and \inlinecode{while} loops or iterators found in many standard libraries rely heavily on the notion of mutation. Also parts of some well known algorithms, like the swap operation in quicksort, can be expressed trivially as mutation.

In practice, almost all programs have some state that determines how the program reacts to input. Real-world examples of state include the location of characters in a game, registered users in an application and cursor position in the buffer when reading bytes from a socket. In the presence of concurrency, when parallel computations are expected to interact with each other, mutability in one form or another is needed to indicate if a computation is still on-going, completed or has encountered an error.


\subsection{Exceptions} \label{effect-types:exceptions}
Another very common effect is the ability to signal about exceptional conditions where the program is unable to compute a result or execute a command. This signaling is achieved by raising an error or exception. An exception could contain information about the condition that caused it, for example malformatted input, and that could possibly be later used when \emph{handling} or recovering from the exception. There are several common reasons why exceptions arise. They usually fall into two categories: technical or logical.~\cite{imprecise-exceptions}

Logical exceptions are usually caused by failing to meet some preconditions regarding the program's state or a function's parameters. A function may have assumptions about its inputs --- a string may need to be in a format that matches a schema in order to parse it successfully, or an integer may need to be positive and less than a certain threshold to represent a year. Sometimes inputs must be compatible with other inputs. An example of this is accessing an array by its index where the accessed index must be less than or equal to the size of the array, or attempting to access authorized content before proper authentication and authorization process.

Technical exceptions are usually related to IO, external events, the runtime environment, or the programming language itself. They can further be divided into synchronous and asynchronous exceptions. Peyton Jones describes synchronous exceptions as something that "arise as a direct result of some piece of code"~\cite{akward-squad}.  On the other hand, asynchronous exceptions are caused by external events and they cannot always be tied to the execution of a particular line of code. In some way, logical and synchronous exceptions are expected exceptions, and asynchronous exceptions are unexpected.

Many synchronous exceptions are related to IO. If attempting to interact with a file that does not exist or the current permissions are not sufficient, the result will likely be an exception of some sort. A significant source of exceptions is communicating over the network with a remote party. Everything from name resolution, routing, transport protocol or communication schema could go wrong. A remote component in a distributed system could be completely unavailable due to a network error or an internal error in that specific component. IO problems also arise when trying to perform an action before initialization, for example via a database connection, file descriptor or IO port.

Other synchronous exceptions may be caused by division by zero or a non-exhaustive pattern match. Probably the most well known synchronous exception is the infamous null reference error, where the program is trying to dereference a pointer that does not point to a valid memory location. In languages that support direct memory access, an attempt to access memory outside of the allowed memory range leads to a program or operating system level exception.~\cite{akward-squad}

Asynchronous exceptions usually originate from the runtime environment of the program, operating system, concurrency, or user interruption. Asynchronously raised exceptions are characterized by the fact that they could occur at an arbitrary point in time~\cite{async-exc}. An example of this is a situation where a thread interrupts the execution of another thread. The whole program could also be interrupted by a user (for example by pressing Ctrl+C) or the runtime, possibly due to a critical error in the program or operating system. Resource exhaustion is another common cause of asynchronous exceptions. Errors like stack overflow or out of memory can happen every time new memory is required from the stack or heap, thus those are categorized as asynchronous. Many environments also support dependencies to libraries that are loaded/linked  dynamically at run time. The programmer cannot always specify the exact time when dynamic loading should take place, and for this reason failing to load required dependencies could be considered an asynchronous exception.~\cite{akward-squad}

Exceptions can also encode another related and important concept, \emph{optionality}. Encoding optionality via exceptions is achieved by raising an exception that contains only a value of the unit type\footnote{A type whose cardinality is 1 (i.e., that has only one value) and thus does not contain any information.}, signaling that no result could be computed and there is no additional information about the exception. Optionality is an approriate choice instead of exceptions when the cause of the exception is trivial. Such cases include unsuccessfully querying a row from a database with specific id, searching an element from an array or trying to find a substring from a string.

Usually, the semantics of raising and handling an error are to interrupt the normal control flow of the program and transfer the execution to the closest appropriate \emph{exception handler}. An exception handler decides if and how to continue the execution, or whether to let the exception bubble up the layers of exception handlers. This "short-circuiting" semantics is a natural way to think and program in the presence of errors. However, the ability to raise errors from an arbitrary location can make it difficult to understand the meaning of the program and prove its correctness. It is a challenge to ensure that all exceptions that may be raised are handled appropriately. Lazy evaluation complicates things even further. The evaluation order in a lazily evaluated language may not be obvious to the programmer. This makes it harder to define clear semantics for exceptions.~\cite{imprecise-exceptions}

Effective and thorough exception handling is one of the most important practices in successful software engineering. Conversely, the inability to do so is one of the most significant factors that causes bugs and failures in software systems. A 2014 study by the University of Toronto studied multiple popular open source distributed software systems, such as Redis, Hadoop and Cassandra and found that a large portion (35\%) of catastrophical failures were caused by trivial mistakes in error handling code. Such mistakes include practices like omitting error handling code completely and writing a TODO-comment instead. In addition to failures, inadequate error handling may expose security vulnerabilities in the system.~\cite{simple-testing-failures}


\subsection{IO}
Programs need the ability to interact with the external world, i.e., with a user, other programs, or devices and sensors. IO is the medium to carry out these interactions. Like interaction in general, IO is often bidirectional --- the term IO is a shorthand for input and output. Input is the ability to observe changes and to receive information from other parties, output enables the program to cause changes in the environment and to dispatch information to others.

Many IO effects are about interacting with the user. Probably the most well-known and fundamental form of user interaction is to display text and graphics by changing pixels on the screen. Another common type of user interaction is via the console, which consists of printing characters to standard output and reading user input from standard input. The use of external devices such as playing sounds from speakers, recording sound from a microphone, or receiving user input from the keyboard, mouse, and touchscreen, is essential in user interaction.

In addition to user interaction, a program can also use devices for other purposes. For example, reading the time from the system clock, requesting the current temperature from a sensor, or setting a digital output to 1 or 0.

Often programs need the ability to store data that persists even when the program is restarted. This is achieved by using a device that allows reading from and writing to a non-volatile memory, such as a hard drive or memory card. Usually an operating system abstracts this persistent data store by providing a file system. However, many embedded devices still communicate directly with persistent memory devices.

The reason for a program to exist is to eventually have an effect on the surrounding world. As IO is the only way to achieve this, it fundamentally distinguishes IO from other effects. Where other effects might be useful for structuring computations and expressing computations in certain ways, IO is \emph{the reason} for programs to exist in the first place~\cite{akward-squad}. To put it the other way around, it would be impossible to detect if a program is running or not if it would not be interacting with its environment.



\section{Concurrency}
Computer programs should be able to run multiple workflows interleaved (concurrency) or at the same time (parallelism). Programs often have many simultaneous users, all of whom should be able to use the program independently of each other. Also, it is characteristic of IO that a large portion of time is spent waiting for a response, rather than calculating results with local computing resources, mainly CPUs. When several operations can be performed in parallel performance improves and the underlying hardware is utilized efficiently.


\subsection{Concurrency adds complexity}
Often workflows must interact with other concurrent workflows. A parent workflow might spawn multiple child workflows and split a task between them. In some situations, one might run several workflows in parallel and choose the result that is computed the fastest, discarding all other results. Concurrent workflows sometimes must use a shared resource, like mutable memory, file, or database connection.

At first glance, these interactions may seem not particularly problematic, but concurrency complicates programs significantly. By default the execution order of concurrent workflows is nondeterministic, because of how tasks are scheduled (usually by the operating system) to run on actual hardware. Many statements in a programming language are compiled to or interpreted as several CPU instructions, executed sequentially at different clock cycles. A canonical example of this is the increment operator (\inlinecode{++} or \inlinecode{+=}), that first reads a variable's value with one instruction and then sets it to a new value with another. If another parallel workflow is updating the same variable at the same time, it might see the value between the two instructions, even though that is rarely the desired behavior. These kind of situations are called \emph{race conditions}.

In order to prevent race conditions, explicit countermeasures are required. One way is to \emph{synchronize} access to shared resources. Usually this means that before a workflow can enter a section of the program or use a shared resource, it must acquire an exclusive \emph{lock}. This means that only one of the workflows has access to the resource at given point of time. Another way, applicable to shared memory, is to use atomic compare-and-swap operations~\cite{concurrent-queue-algorithms}, which enable to observe and update some data, succeeding only if the data was not modified by another workflow in between.

Locks and atomic references are examples of low-level tools for managing concurrency. Each tool comes with a set of trade-offs. When a workflow must acquire multiple locks, a possibility for \emph{deadlock} arises. Deadlock is a situation where concurrent workflows are blocked by each other and neither can continue until the other releases a lock they are holding. Atomic operations can fail, and the failure must be handled for example by retrying until the operation succeeds. Basic atomic operations usually cannot guarantee the atomicity of operations spanning over many atomic references.


\subsection{Concurrency primitives}
In practice concurrency can be implemented with many different constructs. The lowest-level construct commonly accessible to programming languages is a \textit{thread}. It is an OS level abstraction for concurrent execution. Each thread has its own stack, instruction pointer and CPU register values. All threads created by a single process share the same memory space, i.e. they are able to read from and write to the same shared memory blocks.

Threads are run on the actual hardware of the computer; a modern multi-core CPU can execute several workflows in parallel. The OS \textit{schedules} different threads for execution, and after a thread has been executing for a scheduled amount of time, the operating system interrupts the execution and switches the execution to a different thread. The operation where a CPU core switches the execution from one thread to another is called \textit{context switch}. Context switch requires that CPU registers and stack of the previous thread are saved, and respectively registers and stack of the new thread is loaded to the CPU.

% Threads
Traditionally a thread has been the concurrency primitive to turn to when some form of parallelism is required. Threads, however, are not a lightweight construct and they can exist only in limited numbers, usually in the thousands. Context switching between threads involves a significant amount of work, and thus causes performance overhead. Often a context switch defeats many optimizations in contemporary CPUs like caching, pipelining and speculative execution, which in turn amplifies the performance overhead. The issue manifests itself particularly in highly concurrent scenarios where computations are IO bound, which is usually the case in web and enterprise applications.

% Thread pools
A common way to constrain the total number of threads and increase their reuse is to collect multiple threads into a \emph{pool} of threads, where tasks could be submitted for execution instead of operating with individual threads. Once a task is submitted to a thread pool, it is queued and run once there are available threads. This way many concurrent workflows could be \emph{multiplexed} into a smaller number of physical threads. When fewer threads are created and reused by multiple concurrent workflows, the intent is to decrease the number of context switches in the hope of performance gains.

% Cooperative scheduling
Thread pools do not solve the problem that when a thread is waiting an IO operation or another thread to complete, the waiting thread is blocked. When a thread blocks, the OS puts it in a waiting state, meaning that its execution is not continued until the event it is waiting on is triggered. The ideal solution would be that no physical threads are blocked, and blocking is only semantic. This is not possible when threads are preemptively scheduled by the OS. A solution is to change the scheduling model from preemptive to \emph{cooperative}. Cooperative scheduling means that when a workflow is about to be blocked, it will register itself to be scheduled once all of its dependencies are met, and yield the control to other workflows. In this model no physical threads need to be blocked.

% Event loops
Cooperative scheduling is usually implemented by a runtime environment, programming language, or library, that runs on top of preemptively scheduled OS threads. Event loop is a common pattern for implementing cooperative scheduling, and it is used extensively in asynchronous IO or single-threaded environments like JavaScript. The idea is to have a queue that contains computations waiting to get executed, and the event loop picks up and executes tasks from the queue one at a time. Once a task is about to do a blocking operation, it registers a callback. The callback is invoked when the blocking operation completes, and it will add another task to the queue that represents the remaining of the workflow.

% Fibers
Another way to implement cooperative scheduling is \textit{fibers}. They are lightweight threads that are managed and scheduled in the application instead of OS. Each fiber contains a stack and possibly error handlers or thread-local variables, similar to a thread. Fibers require a scheduler that determines what fibers to execute on actual OS threads. The scheduler can multiplex many fibers to run on smaller number of physical threads. Fibers could exist in the hundreds of thousands or millions, and switching execution from one fiber to another is very cheap in comparison to a context switch between threads. The fiber scheduler can assign a fiber to execute on a specific CPU core, which will make it easier to reap benefits from CPU optimizations like caches.


\subsection{Structured concurrency}
When parent workflow spawns many child workflows, it is common that if one of the children encounters an error, the result cannot be computed at all and thus the results of other sibling workflows are not needed anymore. A similar situation may occur with racing workflows: when the first workflow successfully computes a result, the results of other workflows are no longer needed. In both of these situations it would be ideal to cancel the workflows whose results are not needed, to preserve compute resources and make sure that no concurrent workflows remain in execution. Traditional concurrency primitives, such as threads, do not offer this kind of control out of the box.

A solution to this is \textit{structured concurrency}~\cite{structured-concurrency, go-statement-considered-harmful}, which makes it possible to define clear semantics on if and how a child workflow could outlive its parent. The basic idea of structured concurrency is that there is a way to govern how child workflows are handled when the parent workflow completes (by succeeding or failing), or when there is an error encountered in any of the sibling workflows. For example, for a parent workflow that spawns child workflows one would like to define whether the children should be awaited, cancelled, or left orphaned when the parent completes or is cancelled before the children are finished.

Native support for structured concurrency in programming languages is still quite rare, but it has been added to programming languages at an accelerating pace. Kotlin added structured concurrency back in 2018~\cite{kotlin-sc}, Swift 5.5 in 2021~\cite{swift-sc} and Java 19 in 2022~\cite{java-sc}. The feature will probably find its way into more programming languages in the future.

It is difficult to write correct and concurrent programs. Knowledge of concurrency primitives and tools, such as different data structures and CPU instructions, and possible concurrency hazards are required. One has to be especially careful about race conditions, which are not always obvious. Ideally, concurrency could be implemented with high-level code, using operations that take into consideration possible concurrency issues, and deal with low-level details.



\section{Scala} \label{scala}
Scala~\cite{scala-lang} is a high level, statically typed, compiled, and garbage collected programming language, that is both functional and object-oriented. It is eagerly evaluated by default, but supports also lazy evaluation. The first release was in 2004 and the latest version is 3, which was released in 2021. Version 3 is exclusively used in this thesis. The initial and current lead designer of the language is Martin Odersky, a professor at the École polytechnique fédérale de Lausanne. Scala's roots are thus in academia, but its approach is pragmatic.

Scala is most commonly run on the Java Virtual Machine (JVM), but also JavaScript and native code are supported compilation targets. When running on the JVM it is possible to use Java code directly from Scala. The Scala standard library even contains functions for converting Java data types to their Scala counterparts. This gives access to a huge number of Java libraries.

Scala aims to blend the Functional programming (FP) and Object-oriented programming (OOP) paradigms and as a result has features from both. Many OOP concepts like classes, objects, interfaces and subtype polymorphism are supported. In fact, every value in Scala is an object. Scala uses class-based objects with attributes and methods, and supports multiple inheritance. Scala supports generics with lower and upper subtype constraints as well as declaration-site variance. The language also includes many imperative constructs, like loops and mutable variables, that are commonly used in other OO-languages. Perhaps less common in OO-languages, in Scala everything is an expression, including control structures like \inlinecode{if/else}, \inlinecode{try/catch}, and loops. \refsource{scala:basics} demonstrates the basic syntax of Scala.

\begin{algorithm}

\begin{minted}{scala}
val numberZio  = ZIO.succeed(2)
val doubledZio = numberZio.map(n => n * 2)

val sum: UIO[Int] =
  for
    a <- numberZio
    b <- doubledZio
  yield a + b
\end{minted}

\caption{ZIO Basics \label{zio:basics}}
\end{algorithm}

Due to Scala's object-oriented nature, every object is part of a type hierarchy. On top of the hierarchy is \inlinecode{Any}, which is the supertype of all other types. Below \inlinecode{Any} is \inlinecode{Matchable}, which marks values suitable for pattern matching. \inlinecode{Matchable} has two subtypes: \inlinecode{AnyVal}, a supertype for all value types, and \inlinecode{AnyRef}, a supertype for all reference types. \inlinecode{Null} is a subtype of all reference types, except when \emph{explicit nulls} -feature is enabled and \inlinecode{Null} becomes a subtype of \inlinecode{Any}. Scala also has a bottom type \inlinecode{Nothing} that is a subtype of every type. No values of type \inlinecode{Nothing} can ever exist at runtime so the type reflects the absence of a value, for example in the case of infinite recursion or loop, or when the expression throws an exception. The diagram in Figure \ref{fig:scala-type-hierarchy} depicts the type hierarchy.

\begin{figure}
    \centering
    \includegraphics[width=\textwidth]{images/type-hierarchy}
    \caption{Scala 3 type hierarchy.}
    \label{fig:scala-type-hierarchy}
\end{figure}

Variance defines the rules on how the subtype relationships between parameterized types are dependent on the subtype relationship on the type on which it is parameterized. Variance has three variants: \emph{invariance}, \emph{covariance}, and \emph{contravariance}. Invariance means that subtyping relationships present in type parameters are not applied to the parameterized type at all. Covariance states that the subtype relationship of the parameterized type is in the same direction as a type parameter's subtype relationship. Contravariance reverses the subtype relationships between parameterized types and their type parameters. When \inlinecode{Sub} is a subtype of \inlinecode{Super} and \inlinescala{F[_]} is any parameterized type, then
\begin{itemize}
    \item Under covariance, \inlinescala{F[Sub]} is a subtype of \inlinescala{F[Super]};
    \item Under contravariance, \inlinescala{F[Super]} is a subtype of \inlinescala{F[Sub]}; and
    \item Under invariance, \inlinescala{F[Sub]} and \inlinescala{F[Super]} have no subtyping relationship.
\end{itemize}

Covariance is applicable in parameterized types that contain, store, or produce values, in other words the type parameter is in covariant position. Contravariance is applicable in the opposite situation, where values of the type parameter are consumed, i.e. the type parameter appears in a function parameter list, and is said to be in contravariant position. Invariance is useful in situations where it does not make sense for the parameterized type to have inheritance based on type parameter, or when the parameterized type is a mutable, or the type parameter appears both in covariant and contravariant positions. A parametric type with multiple type parameters could declare each type parameter with different variance. For example functions in Scala are contravariant in their input type(s) and covariant in their result type.

An infamous example of a mutable covariant type is the primitive array type in Java and C\#.
These arrays must perform a runtime type check when adding elements to the array, and throw an exception if the type of the element is not compatible with the array, as demonstrated in \refsource{mutable-covariance}. To avoid these mandatory costs and checks, mutable collections in Scala are invariant. Immutable collections and containers, such as \inlinecode{Option} or \inlinecode{Either} are covariant in Scala.

\begin{algorithm}

\begin{minted}{java}
String[] a = new String[1];	
Object[] b = a; // String is a subtype of Object, so this is legal
b[0] = 1; // Runtime exception since cannot add Integer to String[]
\end{minted}

\caption{Covariance in mutable types, like Java primitive array, is problematic \label{mutable-covariance}}
\end{algorithm}

Programming languages differ in the way variance annotations are defined and used. Variance annotations in C\# and Scala are with the parameterized type. On the other hand, in Java one defines variance only when using a parameterized type. The former is called \emph{declaration-site} variance, demonstrated in \refsource{declaration-site-variance} and the latter is called \emph{use-site} variance, demonstrated in \refsource{use-site-variance}. Approaches deviating from these exist, for example TypeScript tries to infer variance, but it has optional declaration-site annotations from version 4.7 onwards. Kotlin has declaration-site variance by default but it emulates some parts of use-site variance with type projections.

\begin{algorithm}

\begin{minted}{scala}
class Invariant[A]      // Invariance is the default
class Covariant[+A]     // Covariance denoted with +
class Contravariant[-A] // Contravariance denoted with -
\end{minted}

\caption{Scala uses declaration-site variance, where the variance of a parameterized type is denoted in its type definition  \label{declaration-site-variance}}
\end{algorithm}

\begin{algorithm}

\begin{minted}{java}
interface Supertype {}
interface Subtype extends Supertype {}

void invariant(List<Supertype> list) {
    /* Get and set list values */
}
void covariant(List<? extends Supertype> list) {
    /* Only get list values */
}
void contravariant(List<? super Subtype> list) {
    /* Only set list values */
}
\end{minted}

\caption{Java has use-site variance, where the desired variance is declared when using the parameterized type. \label{use-site-variance}}
\end{algorithm}

Invariance is the default in Scala and it does not require an explicit annotation. Covariance is declared with a \inlinecode{+} sign before each type parameter. Since contravariance could be seen as the opposite of covariance, it is denoted with a \inlinecode{-} sign.

Many features and principles from functional programming are not only available, but also encouraged in Scala. Pattern matching, first-class functions (\refsource{scala:lambdas}), and tail recursion are all supported and heavily utilized in idiomatic Scala programs. Immutable variables, collections and data-structures are the default way of writing Scala, even though mutable counterparts are also available. Functional data modeling is achieved with the use algebraic data types built into the language. Even though Scala embraces functional programming and imperative code is generally discouraged, introducing arbitrary side effects is possible.

\begin{algorithm}
\begin{minted}{scala}
val ns = List(1, 2, 3)

val mapped1 = ns.map(n => n + 1)
val mapped2 = ns.map(_ + 1) // Same as above in shorter form

val sum1 = ns.foldLeft(0)((x, y) => x + y)
val sum2 = ns.foldLeft(0)(_ + _) // Same as above in shorter form
\end{minted}

\caption{Long and short form of anonymous functions in Scala \label{scala:lambdas}}
\end{algorithm}

In addition to ordinary functions, Scala has a specific function type called \\\inlinecode{PartialFunction} for representing functions that are not defined for all values of their input types. It is a subtype of the (normal) function type, to which it adds the method \inlinecode{isDefinedAt}, which determines if the function is defined for a given value. \refsource{scala:partial-function} shows how to define and use partial functions.

\begin{algorithm}
\begin{minted}{scala}
val someEvensMultipliedByTen: PartialFunction[Option[Int], Int] = {
  case Some(n) if n % 2 == 0 => n * 10
}

val opts  = List(None, Some(2), None, Some(3), Some(4))
val somes = opts.collect(someEvensMultiplied) // List(20, 40)
\end{minted}

\caption{Partial functions in Scala \label{scala:partial-function}}
\end{algorithm}

Some functional languages, such as Haskell, have a special syntax for monadic computations. Scala also provides this syntactic sugar in a form of \inlinecode{for} comprehensions, demonstrated in \refsource{monad:for-syntax}. For comprehension is compatible with any data type that has \inlinecode{map} and \inlinecode{flatMap} methods defined, such as \inlinecode{Option}, \inlinecode{Either}, and \inlinecode{ZIO} (Chapter \ref{zio}). These required methods can be added to any type by using extension methods.

Extension methods, which allow adding methods to a class separately from its definition, are one of Scala's more advanced features. Other state-of-the-art features of Scala include operator overloading and infix operator- and method syntax, higher kinded and dependent types, type lambdas, as well as powerful meta programming capabilities. Scala 3 introduced more cutting edge features, such as automatic type class derivation and union and intersections types.

Probably the most distinguishing feature in Scala is its system of implicits and other contextual abstractions arising from that. A function can mark some of its parameters as implicit and the compiler will try to figure out that parameter from the enclosing scope by its type without the programmer explicitly passing an argument for that parameter. Originally implicit parameters were introduced to achieve similar behavior as Haskell's type classes. Type classes are introduced in more detail in Appendix \ref{typeclasses}. Implicits can, however, also be used for other purposes, such as implicit conversions, context propagation, extension methods, and proving subtyping relationships between generic type parameters at compile-time.~\cite{tc-as-objects}

Syntactically to use implicits, a function can mark some of its parameters as implicit with the keyword \inlinecode{using}. When the function is called, the compiler tries to find a value marked as implicit, with the keyword \inlinecode{given}, from the enclosing scope. If all requested values are found, they are automatically applied as arguments. If any of the implicit parameters is not found, compilation error is reported. \refsource{scala:implicits} shows the function \inlinecode{summon} that searches for an implicit value by type, demonstrating the definition and use of implicit parameters.

\begin{algorithm}
\begin{minted}{scala}
case class Person(age: Int, name: String)

// Define type class
trait Show[A]:
  extension (a: A) def show: String

// Define type class instance
given Show[Person] with
  extension (a: Person)
    def show: String = s"${a.name} is ${a.age} years old"

// Use the type class
def showAll[A: Show](as: List[A]): List[String] =
  as.map(a => a.show)
\end{minted}

\caption{Implicits could be used to encode type classes \label{scala:typeclasses}}
\end{algorithm}

Another advanced feature utilizing implicit resolution is the ability of the Scala compiler to prove type equality or subtype relationships. The class \inlinescala{=:=[From, To]} is for type equality and \inlinescala{<:<[From, To]} for subtype relationship. Both classes extend a function \inlinescala{From => To}, and can be used to transform types. Types with two type parameters could be used as infix in Scala, for example type equality could be written \inlinescala{A =:= B}. When requesting an implicit parameter of either of the types above, Scala compiler synthesizes an instance if the type relationship holds, otherwise reports a compilation error. The act of proving type relationships is said to be \emph{witnessing}, and a common practice is to name the implicit parameter as \emph{evidence}. The feature is useful, for example, when defining functions that make sense only for specific types, as demonstrated in \refsource{scala:witness}, where only nested \inlinecode{Maybe} types could be flattened.

\begin{algorithm}
\begin{minted}{scala}
enum Maybe[+A]:
  case Just(a: A)
  case Nothing

  def flatten[B](using evidence: A <:< Maybe[B]): Maybe[B] =
    this match
      case Just(a) => evidence(a)
      case Nothing => Nothing

Maybe.Just(Maybe.Just(1)).flatten // Compiles
Maybe.Just(1).flatten // Error: Cannot prove that Int <:< Maybe[B]
\end{minted}

\caption{Scala compiler witnesses sub type relationship by providing implicit evidence \label{scala:witness}}
\end{algorithm}

Scala 2 and 3 are also differentiated by the introduction of intersection and union types in Scala 3. Intersection types are denoted with the \inlinecode{&} symbol and union types with \inlinecode{|}. Intersection \inlinescala{A & B} means that the resulting type has properties of both \inlinecode{A} \textbf{and} \inlinecode{B}. Union is the dual of intersection, and the resulting type of \inlinescala{A | B} is either \inlinecode{A} \textbf{or} \inlinecode{B}.

Intersection types are commutative, idempotent, and have \inlinecode{Any} as the identity element. Commutativity means that the order of types included in the intersection does not matter --- Scala considers permutations equal. Idempotency means that type intersectioned with itself is equal to the type itself. \inlinecode{Any} as the identity element means that the intersection of any type \inlinecode{A} with \inlinecode{Any} is equal to \inlinecode{A}, since all types themselves are subtypes of \inlinecode{Any}. Expressed as code, laws of intersection types can be proved with the Scala compiler:
\begin{itemize}
    \item Commutativity: \inlinescala{summon[(A & B) =:= (B & A)]}
    \item Idempotency: \inlinescala{summon[A =:= (A & A)]}
    \item Identity: \inlinescala{summon[A =:= (A & Any)]}
\end{itemize}

Like intersection types, also union types are commutative, idempotent, and obey the identity laws. The identity element is \inlinecode{Nothing}: the union of any type \inlinecode{A} with \inlinecode{Nothing} is equal to \inlinecode{A}, since there are no values of type \inlinecode{Nothing}. Again expressed as code, the laws of union types proved by the Scala compiler are:
\begin{itemize}
    \item Commutativity: \inlinescala{summon[(A | B) =:= (B | A)]}
    then\item Idempotency: \inlinescala{summon[A =:= (A | A)]}
    \item Identity: \inlinescala{summon[A =:= (A | Nothing)]}
\end{itemize}

\chapter{ZIO} \label{zio}
ZIO is a open-source Scala library/framework for managing effects and building concurrent applications. It is based on monadic effects but also takes influence from algebraic effects and handlers. ZIO aims to provide a pragmatic, purely functional, type safe, easily testable and declarative API for asynchronous and concurrent effectful programming. The ZIO ecosystem~\cite{zio} consists of tens of official and several third-party libraries that include among other things, testing, streaming, logging, caching, JSON-parsing, database and other infrastructure interaction, as well as HTTP servers and clients. Today, ZIO is one of the fastest growing and most used ecosystems in Scala.

The development of ZIO started in 2017 by John De Goes. The first stable version was released in August 2020. At the time of writing this, the most recent version is 2.0.6, released in January 2023. De Goes and Adam Fraser, a core contributor to the project, co-authored a book about ZIO called Zionomicon~\cite{zionomicon}, which is extensively used as a reference in this chapter.

The idea of ZIO is to combine multiple effects into a single monad and thus avoiding the need for monad transformers. The library is built around \inlinescala{ZIO[-R, +E, +A]} monad with three type parameters. \inlinecode{E} and \inlinecode{A} parameters represent the error and success channels, much like in Either monad, although ZIO is capable of describing asynchronous and side-effecting computations unlike Either. The \inlinecode{R} parameter describes the requirements, environment, or context, needed to perform the computation. It is similar to the reader monad, but has some extra capabilities that are introduced later in this chapter. Drastically simplifying, a ZIO computation can be seen as function from environment to either an error or a success value: \inlinescala{R => Either[E, A]}. The idea of ZIO's three type parameters is that it should be possible to encode most, if not all, of the effects in a single monad. 

ZIO focuses heavily on statically verifying the correctness of programs. With regards to error handling, environmental requirements, and dependency injection. The three type parameters of ZIO make it possible to statically check that expected errors are handled and the required environment is provided before the program can be executed. Error handling and the environment parameter are discussed in more detail in upcoming sections.

ZIO provides type aliases for common variants, among others:
\begin{itemize}
    \item \inlinescala{type UIO[A] = ZIO[Any, Nothing, A]} has no requirements and cannot fail
    \item \inlinescala{type IO[E, A] = ZIO[Any, E, A]} has no requirements and can fail with \inlinecode{E}
    \item \inlinescala{type URIO[R, A] = ZIO[R, Nothing, A]} has requirement \inlinescala{R} and cannot fail
\end{itemize}

Since ZIO is a monadic effect system, all computations are values that can be transformed with functions. This makes it easy to implement combinators for modifying ZIO-values, thus changing the behavior of the described computation. ZIO provides numerous built-in combinators for error handling, context management, dependency injection, concurrency, retrying and repeating, scheduling, memoizing, resource management, and more. It is also easy to implement complex custom combinators in terms of existing ones.

ZIO's approach to functional programming is pragmatic, aiming for an easy to learn, even for programmers without prior theoretical knowledge about functional programming concepts. Even though the library has strong theoretical foundations in functional programming, the aim is to not have them surface them in the public API more than required. ZIO constructors use lazy, by-name parameters to delay executing unintentional side effects until the ZIO effect is executed. Using ZIO does not require knowledge of concepts like type classes or monad transformers, even though the former is utilized heavily internally.

Function naming mostly avoids terms originating from category theory, symbolic operators, and naming conventions from Haskell. For example, functions corresponding to Haskell's \inlinecode{sequence}, \inlinecode{traverse}, and \inlinecode{bracket}, are named \inlinecode{collectAll}, \inlinecode{foreach}, and \inlinecode{acquireReleaseWith} in ZIO to make them easier to understand. A naming convention originating from Haskell where effectful combinators, such as \inlinecode{foldM}, \inlinecode{ifM}, and \inlinecode{replicateM}, are suffixed with \inlinecode{M}. The meaning of \inlinecode{M} might not be obvious to newcomers and ZIO aims to make it clearer by naming these combinators as \inlinecode{foldZIO}, \inlinecode{ifZIO}, and \inlinecode{replicateZIO}.
Haskell's convention to suffix the names of combinators that discard their result with \_, e.g. \inlinecode{sequence_} or \inlinecode{traverse_}, is not followed: ZIO names are are named \inlinecode{collecAllDiscard} and \inlinecode{foreachDiscard}.

ZIO also takes advantage of multiple advanced features of Scala to make the API more convinient to use. The implicit system is used to provide context information for tracing, derive type class instances and prove type relationships. Dependent types are used, for example, to destructure nested tuples when zipping together multiple ZIO values. There are several combinators that only make sense with specific success or error types. These operators utilize implicit evidence provided by the Scala compiler to make sure they are used appropriately. An example of such cases are error handling operators that are only applicable with effects that can actually fail. Metaprogramming is utilized for example in dependency injection where the the dependency graph is resolved and constructed at compile time, failing compilation if any of the required dependencies is not provided.

Monadic programming in Scala has traditionally suffered from the lack of type inference due to subtyping, forcing the programmer to explicitly write type annotations.
Prior to ZIO, many functional programming libraries in Scala implemented their monads with invariant type variables, because of the issues related to subtyping and type inference mentioned in section \ref{background:monad:monad-transformers} about monad transformers. Since ZIO does not use monad transformers, it does not suffer from limitations associated with them. ZIO embraces the subtyping and variance of Scala by declaring the error and success types covariant, and the environment type as contravariant. This makes type inference a lot more effective and in practice, explicit type definitions are rarely required when combining ZIO effects with different type parameters.



\section{Basic operators}
One of the most used operators are constructors that create ZIO values. Like every monad, ZIO also has a lifting function \inlinecode{ZIO.succeed}. In addition to lifting pure values, it also enables the lifting of non-fallible side effects to ZIO. For lifting side effects that might throw exceptions, \inlinecode{ZIO.attempt} is used. To create failed ZIO effects functions \inlinecode{ZIO.fail} or \inlinecode{ZIO.die} are commonly used. Error handling is discussed in more detail in Section \ref{zio:error-handling}. Constructors for data types from Scala standard library like \inlinecode{Option} and \inlinecode{Either} exist as well. Usage of the most common ZIO constructors is demonstrated in \refsource{zio:constructors}.

\input{sources/zio/constructors}

Starting from the more simpler operators, are the ones including a single ZIO value. Operator for applying a pure transformation to value inside ZIO is implemented by the \inlinecode{map} function. Operator to discard the value of the ZIO and map it to constant value is function called \inlinecode{as}. Common debugging operator for peeking the value inside ZIO without changing the value is called \inlinecode{tap}. ZIO also has specific \inlinecode{debug} operator that will print the value inside ZIO with the provided prefix. Mentioned operators are demonstrated in \refsource{zio:transform}.

\begin{algorithm}

\begin{minted}{scala}
val one: UIO[Int]        = ZIO.succeed(1)
val two: UIO[Int]        = one.map(_ + 1)
val discardOne: UIO[Int] = one.as(34) // same as map(_ => 34)

one.tap(n => ZIO.succeed(println(s"One: $n")))
one.debug("One") // Same as above, prints "One: "34"
\end{minted}

\caption{Common ZIO transform operators. \label{zio:transform}}
\end{algorithm}

Other common category of operators are the ones combining two ZIO values together. The \inlinecode{flatMap} function present in all monads naturally exists in ZIO as well.
For combining two independent ZIO workflows together, there is a whole family of \textit{zipping} operators. Unlike monadic composition via \inlinecode{flatMap}, when zipping values together the second value cannot use the value produced by the first one. The most simple zipping operator, \inlinecode{zip}, simply runs both ZIOs from left to right and combines their result in a tuple. The \inlinecode{zipWith} allows to supply a function to combine the left and right value into the resulting ZIO. Sometimes a ZIO is only evaluated because of the effect it produces, and its return value is not needed. For these puproses \inlinecode{zipRight} and \inlinecode{zipLeft} operators are useful. These combinators evaluate both ZIOs from left to right, but retain only the return value of the side indicated by the operator name. Right and left zipping combinators also have symbolic aliases, generally quite rare in ZIO, \inlinecode{*>} and \inlinecode{<*}, where the arrow points to the side whose value is returned. The combinators for two ZIOs are demonstrated in \refsource{zio:binary-combinators}.

\begin{algorithm}

\begin{minted}{scala}
val num  = ZIO.succeed(34)
val str  = ZIO.succeed("A string value")
val tell = ZIO.succeed(println("Hello World"))

// All three below are semantically equal
val v1: UIO[(Int, String)] = num.flatMap(n => str.map(s => (n, s)))
val v2: UIO[(Int, String)] = num.zipWith(str)((n, s) => (n, s))
val v3: UIO[(Int, String)] = num.zip(str)

val zipRight: UIO[Int] = tell.zipRight(num)
val zipLeft: UIO[Int]  = num.zipLeft(tell)

// Evaluation order: tell, num, tell. Returns the value of num
val toldTwoTimes: UIO[Int] = tell *> num <* tell
\end{minted}

\caption{Common binary combinators in ZIO. \label{zio:binary-combinators}}
\end{algorithm}

When required to combine more than two ZIOs togheter, for example in a loop-like situation, there are operators for that as well. Effectful for loop is provided by the \inlinecode{ZIO.foreach} function, which takes a collection of values, and a function that performs some effectful computation for each value. The operator performs all computations and returns a collection of results. Similar operator is \inlinecode{collectAll}, which receives a collection of ZIO computations, and returns a collection containing the results of the computations. Both operators are demonstrated in \refsource{zio:multi-combinators}. In order to effectfully fold over a collection of values, ZIO provides, among others, \inlinecode{mergeAll}, \inlinecode{reduceAll}, \inlinecode{foldLeft}, and  \inlinecode{foldRight} functions to compute a single summary value from a collection.

\begin{algorithm}

\begin{minted}{scala}
def findById(id: Int): UIO[Result] = ???
def combineResults(total: Int, result: Result): Int = ???

val ids = List(1, 2, 3)

val found1: UIO[List[Result]] = ZIO.foreach(ids)(findById(_))
val found2: UIO[List[Result]] = ZIO.collectAll(ids.map(findById(_)))

val combined: UIO[Int] =
  ZIO.mergeAll(ids.map(findById(_)))(0)(combineResults(_, _))
\end{minted}

\caption{Common combinators for multiple values in ZIO. \label{zio:multi-combinators}}
\end{algorithm}



\section{Error handling} \label{zio:error-handling}
Proper error handling is essential in any non-trivial application, like mentioned in section \ref{effects:exceptions}. Failures in ZIO are described in a referentially transparent way by returning values that represent the error, instead of throwing exceptions. Like other monads capable of encoding exceptions, ZIO is stops execution of the success channel on first encountered error, until the error is handled with one of the error handling combinators. Much of the errors are tracked in types, making it possible to have static proof that all declared errors are handled. ZIO advocates its error model, which is promised not to lose any errors, even asynchronous, parallel, caused by interruptions, or exceptions thrown by finalizers.

ZIO divides failures into three categories: errors, defects and fatal errors. Fatal errors are thrown by the runtime platform (usually JVM), such as \inlinecode{OutOfMemoryError}, which results in immediate termination of the application, and thus are not very interesting in this context. The two remaining error types describe failures that are possible for the programmer to interact with. Errors are represented as the \inlinecode{E} parameter in ZIO, and are tracked in the types. \inlinecode{Nothing} has a cardinality of zero, which proves that ZIO with \inlinecode{Nothing} in the error channel cannot produce a failing ZIO, thus is infallible. Defects are not reflected in the types, and practically any ZIO can produce a defect when executed. The type of defect is always Java's \inlinecode{Throwable}.

The error channel should be used for business errors that are expected to happen and there is a meaningful way to handle and recover from. On the other hand, defects are failures that are unexpected, or there is no meaningful way to handle or recover from. Because Scala programs are mostly run on the JVM, where exceptions could be thrown anywhere, ZIO runtime catches all thrown exceptions and reports them as defects. This makes it easier to integrate with code not written with ZIO, such as Java-libraries where throwing exceptions is the de-facto error reporting and handling strategy. Roughly speaking, logical exceptions (discussed in section \ref{effects:exceptions}) are usually errors, while technical exceptions are usually defects.

Errors in ZIO are internally represented with a data type \inlinecode{Cause}, which is an algebraic structure called \textit{semiring}, that is capable of capturing the full chain of possible failures, including errors, defects, and interruptions, sequential or parallel. The data type also keeps track of a trace that lead to the failure described by a specific \inlinecode{Cause}. Trace is similar to ordinary stack trace but it is able to describe operations across asynchronous  boundaries and has an option not to expose unnecessary details of the underlying runtime implementation in the trace. ZIO provides operators to interact with the \inlinecode{Cause} data type directly, but usually higher level operators that work with error or defect types are preferred. Definition of simplified \inlinecode{Cause} data type and example of its usage is provided in \refsource{zio:cause}.

\begin{algorithm}

\begin{minted}{scala}
// Cause in ZIO also includes traces omitted here
enum Cause[+E]:
  case Empty
  case Fail(value: E)
  case Die(value: Throwable)
  case Both(left: Cause[E], right: Cause[E])
  case Then(left: Cause[E], right: Cause[E])
  case Interrupt(fiberId: FiberId)

val a = ZIO.dieMessage("A")
val b = ZIO.fail("B").ensuring(ZIO.sleep(5.millis).timeout(1.milli))
a.zipPar(b).cause.debug
// Cause.Both(
//   Cause.Die(java.lang.RuntimeException("A")),
//   Cause.Then(
//     Cause.Fail("B"),
//     Cause.Interrupt(<FiberId of the interrupting fiber>),
//   ),
// )
\end{minted}

\caption{Cause data type captures the full cause of failures \label{zio:cause}}
\end{algorithm}

When two ZIOs are composed together, the composed ZIO could fail either with the error from the first, or the error from the second one. The order in which the error types appear, should not matter and all permutations consisting of same types should be equal, i.e. the composition is commutative. If the two ZIOs share the same error type, the resulting ZIO has equal error type with the original ZIOs, i.e. the composition is idempotent. If either of the two ZIOs cannot fail (the error type is \inlinecode{Nothing}), its error type does not contribute to the resulting error type, i.e. the composition has \inlinecode{Nothing} as the identity element. Union types in Scala 3 naturally have all these properties and precisely expresses the composition of error types. Another way of thinking is to consider the error type as a set of possible error types, composition is set union of their errors and \inlinecode{Nothing} represents empty set. If the execution of the ZIO fails, the error is \textbf{one of} the errors in the set of possible errors. \refsource{zio:error-accumulation} demonstrates the accumulation of errors in types.

\begin{algorithm}

\begin{minted}{scala}
val num1: ZIO[Any, ErrorA, Int]          = ???
val num2: ZIO[Any, ErrorA, Int]          = ???
val num3: ZIO[Any, ErrorB, Int]          = ???
val doSomething: ZIO[Any, Nothing, Unit] = ???

// 'ErrorA' is included only once in the error type
// 'Nothing' is not included at all in the error type
val composed: ZIO[Any, ErrorA | ErrorB, Int] =
  for
    n1 <- num1        // ErrorA
    n2 <- num2        // ErrorA
    _  <- doSomething // Nothing
    n3 <- num3        // ErrorB
  yield n1 + n2 + n3
\end{minted}

\caption{Typed error accumulation when composing multiple ZIO values \label{zio:error-accumulation}}
\end{algorithm}

Ideally there would be no need to explicitly add the type annotation about the error type when composing ZIOs toghether, and simply rely on type inference. Scala compiler tries automatically to \textit{unify} the types, i.e. find the closest common supertype between the composed ZIO values. The \inlinecode{E} parameter in ZIO is covariant, which is essential for type inference when combining multiple ZIOs together. Because \inlinecode{Nothing} is subtype of every type, ZIO that has \inlinecode{Nothing} in the \inlinecode{E} channel is automatically considered to be a subtype of ZIO that has same \inlinecode{R} and  \inlinecode{A} type parameters.

There are many similar operators for working with the values in the error channel as in the success channel. For example \inlinecode{mapError}, \inlinecode{flatMapError} and \inlinecode{tapError} all work similarly to their success channel counterparts. Some of the most common error handling operators include catching some or all errors, providing a fallback computation, or folding over error and success values. Operators \inlinecode{catchAll} and \inlinecode{catchSome} behave like catch blocks in a try-catch clause, and like the names suggest, it's possible to handle either a subset or all errors. The \inlinecode{orElse} operator makes it possible to define a fallback computation whose success and error is used in the case when the original ZIO fails. ZIO has many variations of \inlinecode{fold} for pure and effectful folding which are semantically similar to folding an \inlinecode{Either}, discussed more in section \ref{background:monads:either}). These basic error handling operators are demonstrated \refsource{zio:error-handling-operators}.

\begin{algorithm}

\begin{minted}{scala}
type Error = ErrorA | ErrorB | ErrorC

val mayFail: IO[Error, Int] = ???

val handled: IO[Nothing, Int] = mayFail.catchAll(e => ZIO.succeed(0))

val someHandled: IO[Error, Int] =
  mayFail.catchSome { case _: ErrorA => ZIO.succeed(34) }

val folded: UIO[Int] = mayFail.fold(e => -1, n => n + 10)

val withFallback: IO[Nothing, Int] = mayFail.orElse(ZIO.succeed(0))
\end{minted}

\caption{Basic error handling operators in ZIO. \label{zio:error-handling-operators}}
\end{algorithm}

In addition to \inlinecode{try-catch} like semantics described above, \inlinecode{try-finally} is a common pattern in imperative programming. Regardless whether the code in the \inlinecode{try} throw exceptions or not, the code in \inlinecode{finally} block guaranteed to be executed. The underlying idea is that there is finalizer(s) that need to be run after a certain block of code is executed. ZIO also supports this pattern with several operators that are guaranteed to execute the finalizers even in the presence of parallelism, asynchrony, concurrency, interruption, errors, and defects. \refsource{zio:finalizers} demonstrates the basic finalizing operator \inlinecode{ensuring} that executes the specified finalizer regardless of any kind of failure or interruption. Other, higher level, operators for \inlinecode{try-finally} like semantics are discussed more thoroughly in section \ref{zio:resource-management} about resource management.

\begin{algorithm}

\begin{minted}{scala}
val finalizer = ZIO.succeed(println("Finalizer executed"))

// The finalizer is executed once after each ZIO below is executed
val success: UIO[Int]    = ZIO.succeed(1).ensuring(finalizer)
val error: IO[Int, Int]  = ZIO.fail(42).ensuring(finalizer)
val defect: UIO[Nothing] = ZIO.dieMessage("No").ensuring(finalizer)

val interruption: UIO[Unit] = for
  fiber <- ZIO.sleep(1.second).ensuring(finalizer).fork
  _     <- fiber.interrupt // The finalizer is executed here
yield ()
\end{minted}

\caption{Basic finalizer operator \inlinecode{ensuring} in ZIO. \label{zio:finalizers}}
\end{algorithm}

The fact that ZIO has two typed channels of output values (error and success), makes it possible to create interesting combinators that switch values between the two channels. Operator that simply swaps the channels with each other is \inlinecode{flip}. Another way to expose errors in the success channel is the \inlinecode{either} operator that converts fallible a ZIO to \inlinescala{ZIO[R, Nothing, Either[E, A]]}, resulting in an effect that cannot fail, but instead surfaces errors with \inlinecode{Either} in the success channel. The dual of \inlinecode{either} is the operator \inlinecode{absolve} that separates \inlinecode{Either} cases from the success channel to error and success channels of ZIO. The \inlinecode{Cause} data type could also be exposed in the success channel with the \inlinecode{cause} operator, making it possible to operate errors, defects and interruptions at the same time. The reverse operator is \inlinecode{uncause}, that hides the \inlinecode{Cause} data type from the type signature. Type signatures of mentioned operators can be seen in \refsource{zio:error-tricks}.

\input{sources/zio/error-tricks}

Same exceptions might be considered errors at some abstraction level, and defects at some other abstraction level. For example when implementing \acronym{DAO}{Data Access Object}, that is directly interacting with a relational database, it would be sensible to treat \inlinecode{SQLException} as error and expose it in the \inlinecode{E} parameter. On the other hand, higher level abstractions using the DAO, like repositories or services, usually should not to declare \inlinecode{SQLException} in their signature, and treat it as a defect.

ZIO contains operators for switching values from the error channel to defect channel and the other way round. Simple way to convert errors to defects is to consider all errors as defects, which could be achieved with the \inlinecode{orDie} operator that switches all errors from the error channel to defect channel. In order to have more control of what errors to retain, the \inlinecode{refineOrDie} operators are useful. They allow to pick desired errors by providing a type parameter or a partial function, and the operators converts all errors not matching the type parameter or partial function to defects. To go the other way round and switch values from the defect channel to error channel, \inlinecode{resurrect} operator moves all defects to errors and \inlinecode{unrefine} moves some defects to errors, like \inlinecode{refine} but the other way around. \refsource{zio:defect-handling} demonstrates the usage of these operators.

\begin{algorithm}

\begin{minted}{scala}
val readFile: IO[Throwable, Array[Byte]] =
  ZIO.attempt(new FileInputStream("file.txt").readAllBytes())

val allErrorsToDefects: IO[Nothing, Array[Byte]] = readFile.orDie

val someErrorsToDefects: IO[FileNotFoundException, Array[Byte]] =
  readFile.refineToOrDie[FileNotFoundException]

val allDefectsToErrors: IO[Throwable, Array[Byte]] =
  allErrorsToDefects.resurrect

val someDefectsToFailure: IO[FileNotFoundException, Array[Byte]] =
  allErrorsToDefects.unrefineTo[FileNotFoundException]
\end{minted}

\caption{ZIO operators for switching between errors and failures. \label{zio:defect-handling}}
\end{algorithm}

Sometimes when an error occurs, it can be resolved by retrying the operation that produced the error. Retries in ZIO only apply when the failure is in the error channel, and not in the defect channel. If one would like to retry even when defect happens, it must first be surfaced to the error channel. Probably the simplest retry operator is \inlinecode{eventually}, which will retry forever until the operation succeeds. Usually it makes sense to limit the number of retries, and \inlinecode{retryN} operator enables just that. For specifying a custom rules when to retry and when to give up, ZIO has \inlinecode{retryUntil} and \inlinecode{retryWhile} operators that take a predicate as a parameter and retry according to that predicate. Basic retry operators are demonstrated in \refsource{zio:retry}.

\begin{algorithm}

\begin{minted}{scala}
val readFile: IO[Throwable, Array[Byte]] =
  ZIO.attempt(new FileInputStream("file.txt").readAllBytes())

val retryForever: UIO[Array[Byte]]             = readFile.eventually
val retryFiveTimes: IO[Throwable, Array[Byte]] = readFile.retryN(5)

val retryUnlessFileNotFound: IO[Throwable, Array[Byte]] =
  readFile.retryUntil {
    case _: FileNotFoundException => true
    case _                        => false
  }
\end{minted}

\caption{Basic retry operators in ZIO. \label{zio:retry}}
\end{algorithm}

Instead of immediately retrying, a common way is to schedule the retries with a delay in order to allow the error resolve. ZIO has specific data type for describing retry policies and other scheduling use cases called \inlinecode{Schedule}. It is a purely functional and composable data type capable of describing complicated schedules. In addition to retries, schedules are also applicable for describing the repetition and scheduling the execution of ZIO computations. \refsource{zio:schedule} introduces some basic \inlinecode{Schedule} constructors and combinators. When retrying ZIO with a delay, one might desire to limit the total time the computation can take, which is achieved with the \inlinecode{timeout} operator.

\begin{algorithm}

\begin{minted}{scala}
Schedule.spaced(7.millis) // Constant delay between every computation
Schedule.fixed(7.millis)  // Computations start at constant intervals
Schedule.fibonacci(2.millis)   // 2ms | 4ms | 6ms | 10ms | 16ms
Schedule.exponential(2.millis) // 2ms | 4ms | 8ms | 16ms | 32ms

Schedule.forever   // Schedule always wants to continue
Schedule.stop      // Schedule that never wants to continue
Schedule.recurs(5) // Schedule that wants to continue 5 times

left ++ right // First left schedule to complection, then right
left && right // Recurs when both schedules want to continue
left || right // Recurs when either schedule wants to continue
\end{minted}

\caption{Schedule data type in ZIO \label{zio:schedule}}
\end{algorithm}



\section{Environment}
Arguably the most distinguishing feature about ZIO is its environment or \inlinecode{R} type. The possibility to express environmental/contextual requirements of a computation, plays a big part in the fact that ZIO can encode several effects in one monad, thus mostly eliminating the need for monad transformers. ZIO environment is similar to a reader monad, but has couple of key differences. Unlike reader monad whose only effect is to provide read-only access to some context, the ZIO environment is just a one of the effects that can be expressed with ZIO. Also the environment type composes naturally when combining multiple ZIO values. The environment type in ZIO can be changed from one type to another, similar to indexed reader monads~\cite{monad-factory}. Its also possible to locally both introduce environmental requirements and eliminate some or all environmental requirements.

Recall that a mental model of a \inlinecode{ZIO[R, E, A]} is function \inlinecode{R => Either[E, A]}.
Before a ZIO can be executed the required environment must be provided, just like function must be provided with the parameters before it can be evaluated. A ZIO workflow, that has no environmental requirements, has \inlinecode{Any} as its environment type. Function \inlinecode{f: Any => Either[E, A]} function accepts \textit{anything} as its argument and can be called, for example, by providing the unit value \inlinecode{f(())}, number \inlinecode{f(42)}, or string \inlinecode{f("foo")} as its argument. The analogy applies to ZIO, where \inlinecode{ZIO[Any, E, A]} is ready to be executed without providing any environment.

\begin{algorithm}

\begin{minted}{scala}
val num1: ZIO[String, Nothing, Int] = ???
val num2: ZIO[Int, Nothing, Int]    = ???
val num3: ZIO[Any, Nothing, Int]    = ???

// 'Any' does not appear in the environment type
val composed: ZIO[String & Int, Nothing, Int] =
  for
    n1 <- num1
    n2 <- num2
    n3 <- num3
  yield n1 + n2 + n3
\end{minted}

\caption{Environment types accumulate when composing multiple ZIO values. \label{zio:environment-accumulation}}
\end{algorithm}

When combining ZIO values together, the resulting ZIO naturally has environmental requirements from \textbf{all} combined ZIOs. Similarly to error accumulation, the composition should be commutative and have \inlinecode{Any} as its identity element. Scala 3 intersection types have these properties and thus expresses the composition accurately. \refsource{zio:environment-accumulation} demonstrates the accumulation of environment types when composing ZIO values.

Basic operations for interacting with the environment are adding requirements to it and eliminating all or part of the requirements. It is also possible to translate one environmental requirement to another. A value from the environment can be accessed with \inlinecode{ZIO.service} function, which is really similar to the \inlinecode{ask} function in Reader monad with the exception that \inlinecode{ZIO.service} can return a part of the environment instead of the entire environment. \refsource{zio:environment-access} demonstrates different operators for accessing the environment, which add corresponding environmental requirements.

\begin{algorithm}

\begin{minted}{scala}
// Same as 'ask' in reader monad
val ask: ZIO[String, Nothing, String] =
  ZIO.service[String]

// Eqivalent to: ZIO.service[String].map(_.length)
val askAndMap: ZIO[String, Nothing, Int] =
  ZIO.serviceWith[String](_.length)

// Equivalent to: ZIO.service[Random].flatMap(_.nextInt)
val askAndFlatMap: ZIO[Random, Nothing, Int] =
  ZIO.serviceWithZIO[Random](_.nextInt)
\end{minted}

\caption{Operators for adding requirements or accessing the ZIO environment. \label{zio:environment-access}}
\end{algorithm}


\subsection{Zlayer}
Environmental requirements in ZIO are provided in the form of a purely functional data type called \inlinecode{ZLayer}. \inlinecode{ZLayer[RIn, E, ROut]} has same three type parameters as ZIO itself, and thus is capable of expressing effectful, asynchronous, and possibly failing construction of requirements. The \inlinecode{RIn} parameter in \inlinecode{ZLayer} represents dependencies that are required in order to construct value of type \inlinecode{ROut}. These dependencies between layers form a graph of dependencies, like demonstrated in \refsource{zio:zlayer-graph}.

\begin{algorithm}

\begin{minipage}{0.70\textwidth}
\begin{minted}{scala}
val layerA: ZLayer[Any,   Nothing, A] = ???
val layerB: ZLayer[A,     Nothing, B] = ???
val layerC: ZLayer[A,     Nothing, C] = ???
val layerD: ZLayer[B & C, Nothing, D] = ???
\end{minted}
\end{minipage}
%
%
\begin{minipage}{0.20\textwidth}
\begin{tikzpicture}
     \node (A) at (90:1)  {A};
     \node (B) at (180:1) {B};
     \node (C) at (0:1)   {C};
     \node (D) at (270:1) {D};
    
      \path[->] (B) edge (A)
                (C) edge (A)
                (D) edge (B)
                    edge (C);
\end{tikzpicture}
\end{minipage}

\caption{Dependencies between \inlinecode{ZLayer}s form a graph. \label{zio:zlayer-graph}}
\end{algorithm}



Environment for a ZIO workflow is provided with operators such as \inlinecode{provide} (provide all requirements), \inlinecode{provideSome} (provide a part of requirements), and \inlinecode{provideLayer} (convert existing requirements into other requirements), that take \inlinecode{ZLayer}(s) as their argument. Also the \inlinecode{apply} method in \inlinecode{ZLayer} can be used to eliminate requirements from a ZIO workflow. ZIO can resolve the dependency graph with compiler macros, for example in \inlinecode{ZIO.provide} and \inlinecode{ZLayer.make} functions, and raise a compilation error if all required dependencies are not provided. Different ways of providing layers is demonstrated in \refsource{zio:zlayer-provide}.

\begin{algorithm}

\begin{minted}{scala}
// ZLayer.make and ZIO.provide resolve the dependency graph
val useD: ZIO[D, Nothing, Int] = ZIO.service[D].as(34)

val layer: ZLayer[Any, Nothing, D] =
  ZLayer.make[D](layerA, layerB, layerC, layerD)

val provided1: ZIO[Any, Nothing, Int] = useD.provideLayer(layer)
val provided2: ZIO[Any, Nothing, Int] = layer(useD) // layer.apply
val provided3: ZIO[Any, Nothing, Int] =
  useD.provide(layerA, layerB, layerC, layerD)
\end{minted}

\caption{Providing layers from \refsource{zio:zlayer-graph} to a ZIO. \label{zio:zlayer-provide}}
\end{algorithm}

\inlinecode{ZLayer}s along with ZIO environment are the basis of dependency injection in ZIO. Dependency injection in ZIO is resolved statically at compile time, so program with missing dependencies won't compile. Although there are several conventions and patterns related to dependency injection in ZIO programs, they are not discussed in more detail in this thesis.
\todo{Onko tarpeellinen/sopiva maininta?} 


\subsection{ZEnvironment}
Example in \refsource{zio:environment-accumulation} had ZIO value \inlinecode{composed} that has \inlinescala{String & Int} as the environment type. It is not possible for the type to exists at runtime, since there is no type that is both \inlinecode{String} and \inlinecode{Int}, so the type is only sensible at compile time. These kind of types that only exist at compile time are sometimes called \textit{phantom types}~\cite{fun-phantom-types}.

The \inlinecode{R} type parameter in ZIO is a phantom type, and therefore represents the required types only at compile time. However, every type present in the environment type intersection must have a corresponding value at runtime. This is achieved with a data type called \inlinecode{ZEnvironment[R]}, which is can be seen as a map associating every type in the environment type intersection to a value, as demonstrated in \refsource{zio:zenvironment}.

\begin{algorithm}

\begin{minted}{scala}
// Can be thought of as: Map(Int -> 42, String -> "foo")
val environment: ZEnvironment[String & Int] =
  ZEnvironment.empty
    .add[Int](42)       // Explicit types here are not required
    .add[String]("foo") // but they are added for clarity

// Values from the environment can be accessed by their type
val int    = environment.get[Int]    // 42
val string = environment.get[String] // "foo"
\end{minted}

\caption{\inlinecode{ZEnvironment} contains the required environment for ZIO workflow. \label{zio:zenvironment}}
\end{algorithm}

With this knowledge, the mental model of ZIO can be updated to be \inlinecode{ZEnvironment[R] => Either[E, A]}. Since \inlinecode{ZEnvironment} is a low-level data type used internally to represent the environmental requirements of ZIO, it's not advised to use it directly, but instead use higher-level operators and data types such as \inlinecode{ZLayer}.


\subsection{Use cases}
The ZIO environment can be used in many ways. In addition to providing read-only data to computations like reader monad, it can be utilized in other interesting ways as well. It can describe mutable state, safe resource management (discussed more in Section \ref{zio:resource-management}), or dependency-injection. One could also use environmental requirement as a marker that certain ZIO computation must be run in a specific context. Another common use-case is to define combinators that translate a certain environmental requirement into another.

Probably the most basic use-case is to provide some static data/context, which the computation can use as it wishes. Example of such data is configuration data in a web application, possibly containing a URL for performing http requests. This is demonstrated in \refsource{zio:environment-simple}.

\begin{algorithm}

\begin{minted}{scala}
case class Configuration(url: String)

val useConfiguration: ZIO[Configuration, Nothing, Result] =
  ZIO.serviceWithZIO[Configuration](conf => makeRequest(conf.url))

val configurationLayer: ZLayer[Any, Nothing, Configuration] =
  ZLayer.succeed(Configuration(url = "https://example.com"))

val configurationProvided: ZIO[Any, Nothing, Result] =
  configurationLayer(useConfiguration)
\end{minted}

\caption{Static data can be provided to computations with the ZIO environment. \label{zio:environment-simple}}
\end{algorithm}

Another use case is to encode mutable state in the environment, similar to monad transformer for \inlinecode{State} monad. This is achieved with data type describing mutable references evaluated in the ZIO monad, such as \inlinecode{Ref} or \inlinecode{ZState} which is purposefully built for this use case. State can be accessed with \inlinecode{ZIO.getState} function, which also adds a state requirement to the environment. State requirement can be eliminated using the \inlinescala{ZIO.stateful} operator by providing the initial state. \refsource{zio:state} demonstrates the usage of these operators in stateful computation. A nice byproduct of encoding state in the ZIO environment, is that the environment can carry several different states at the same time, as long as the states are of different types.

\begin{algorithm}

\begin{minted}{scala}
val statefulComputation: URIO[ZState[Int], Int] = for
  state <- ZIO.getState[Int] // Access state
  _     <- ZIO.setState(state + 1) // Modify state
yield state

val statefulProgram: URIO[ZState[Int], Unit] = for
  _ <- statefulComputation.debug("First state")
  _ <- statefulComputation.debug("Second state")
  _ <- ZIO.getState[Int].debug("Final state")
yield ()

// Provide initial state (0) to the stateful computation
// When executed prints:
// "First state: 0", "Second state: 1", "Final state: 2"
val stateRequirementProvided: URIO[Any, Unit] =
  ZIO.stateful(0)(statefulProgram)
\end{minted}

\caption{Mutable state can be encoded with the environment in ZIO. \label{zio:state}}
\end{algorithm}

Environmental requirements can be converted from one type to another by eliminating one requirement and adding a new one. \refsource{zio:user-session} demonstrates one such situation. In that example \inlinecode{businessLogic} requires a \inlinecode{UserSession} from the environment. There is a \inlinecode{UserService} that can validate a token (\inlinecode{String} in this case) and succeed with \inlinecode{UserSession}, or fail validation with \inlinecode{TokenError}. For example, in the context of a web application, a token could be extracted from the http request. The helper function \inlinecode{UserService.withSessionFromToken} takes two parameters: a token and a ZIO computation that requires \inlinecode{UserSession} from the environment, and returns a ZIO computation that requires \inlinecode{UserService} from the environment, which will be used to validate the token. If the validation is successful a \inlinecode{UserSession} is provided to the computation. If validating the token fails, it fails the whole computation with \inlinecode{TokenError} and the computation received as a parameter will not be executed. The possibility that validating the token might fail can be observed from the fact that  \inlinecode{TokenError} is added to the error type of the returned ZIO computation.

\begin{algorithm}

\begin{minted}{scala}
trait UserService:
  def validate(token: String): IO[TokenError, UserSession]

object UserService:
  def withSessionFromToken[R: Tag, E, A](token: String)(
      needsSession: ZIO[R & UserSession, E, A],
  ): ZIO[R & UserService, E | TokenError, A] =
    val session = ZIO.serviceWithZIO[UserService](_.validate(token))
    val layer   = ZLayer(session) // Create a ZLayer from ZIO value
    layer(needsSession) // Provide session as layer to ZIO workflow

val businessLogic: ZIO[UserSession, Nothing, Result] = ???

val program: ZIO[UserService, TokenError, Result] = for
  token <- getToken // For example from a HTTP request
  result <- UserService.withSessionFromToken(token) {
    businessLogic
  }
yield result
\end{minted}

\caption{ZIO environment can be used to translate a contextual requirement to other requirement. \label{zio:user-session}}
\end{algorithm}

The power of the environment type comes from the fact that it supports many different overlapping use cases. For example, configuration, state, and sessions can coexist in the environment without interfering with each other. The environment can be provided locally to a specific computation or globally to the entire program.


\subsection{Similarity to algebraic effects}
It may not be immediately obvious how ZIO is similar to algebraic effects and handlers.
However, if we consider that each type in the environment intersection represents a specific effect, adding or interacting with environmental requirements represents an effectful operation, and removing environmental requirement with \inlinecode{ZLayer} represents handling an effect, the similarity is imminent. 
\refsource{zio:zlayer-eff-handler} 

Like handlers in algebraic effects, \inlinecode{ZLayer}s can handle (or discharge) the effect by removing it altogether, or it can translate one effect into another. Similar to handlers in algebraic effects, \inlinecode{ZLayer}s commonly form a graph of dependencies between other \inlinecode{ZLayer}s. ZIO environment composes in similar way as effects in language that natively supports algebraic effects and handlers, such as Unison.

With \inlinecode{ZLayer}s it is possible to define polymorphic handler, which only handles a subset of all effects in a specific expression. In practice this means that a \inlinecode{ZLayer} eliminates only a part of the environment, while leaving the rest in place. \refsource{zio:zlayer-eff-handler} demonstrates the mentioned similarity and polymorphic handlers.

\begin{algorithm}

\begin{minted}{scala}
trait ZLayer[-RIn, +E, +ROut]:
  // Environmental requirement of type ROut is removed, and RIn is
  // added to the ZIO received as parameter.
  def apply[R, E1, A](
      zio: ZIO[ROut & R, E1, A]
  ): ZIO[RIn & R, E1 | E, A]

val handleAtoB: ZLayer[B, Nothing, A] = ??? // Changes A -> B
val handleB: ZLayer[Any, Nothing, B]  = ??? // Eliminates B

val effect: ZIO[A & Boolean, Nothing, Int] = ???

// ZLayer is polymorphic in the type of environmental requirement
// Here it removes B, adds A, and leaves Boolean as is
val handledA: ZIO[B & Boolean, Nothing, Int] = handleAtoB(effect)

// ZLayer (handler) for B does not have any requirements, so B is
// removed from the environment entirely, leaving only Boolean
val handledB: ZIO[Boolean, Nothing, Int] = handleB(handledA)
\end{minted}

\caption{ZIO workflows and ZLayers can be seen as really similar to algebraic effects and handlers. \label{zio:zlayer-eff-handler}}
\end{algorithm}

Effect polymorphism (demonstrated in Listings \ref{scala:cc-eff-polymorphism}, \ref{alg-eff:polymorphism-unison}, and \ref{alg-eff:polymorphism-koka}) is limited since every ZIO computation is evaluated in monadic context, however. Also handlers in ZIO are not as expressive, since they do not receive a continuation to the rest of the program like algebraic effects handlers.


\section{Resource management} \label{zio:resource-management}
At a high level, resource management consists of three parts: acquiring resources, using resources and releasing resources after they are used and no longer needed. Countless things can be viewed as resources that need to be acquired and released: concurrency or database locks, allocated memory, open file handles or network sockets, connections from a connection pool, or spawned processes/threads. Even a database transaction is special kind of resource where releasing it either commits the transaction or rolls it back.

The important part is that once a resource is acquired, it must be released, even if using the resource raises an exception or fails in any other way. This behavior is can be described with contextual data type that is added to the environment when the resources are acquired, and stays in the environment as long as there are resources that need to be released. A consequence of this is that acquired resources are visible in the type signature, and the compiler is able to help in making sure that acquired resources are actually released.

Safe resource management in ZIO relies on information threaded through computations in the ZIO environment. In ZIO, the data type describing lifetime of resources is called \inlinecode{Scope}. In principle, a \inlinecode{Scope} is very simple and has only two operations: one to add a finalizer that is executed when the scope is closed, and one to actually close the scope. Computation that acquires a resource, requires that a \inlinecode{Scope} is in the environment. After the resource is acquired, a finalizer for releasing the resource is added to the scope. Before the ZIO is executed, the \inlinecode{Scope} must be provided. The provided \inlinecode{Scope} determines how long the resource is usable and when it is released. 

To create a resource, ZIO has \inlinecode{acquireRelease} constructor and several variants for it. Like the name suggests, these constructors take two ZIO computations as their parameter: one to acquire the resource and one to release it. They return a ZIO computation that succeeds with the resource, and have added \inlinecode{Scope} to the environment. In order to determine the extent of a \inlinecode{Scope} and remove it from the environment, ZIO provides a operator called \inlinecode{scoped}. It takes a ZIO computation with requirement to scope as a argument, provides the scope, runs the computation with the scope and finally closes the scope. Several resourceful ZIOs could be interpreted in a different ways depending how the \inlinecode{Scope} is provided, which changes the order of acquire and release, in other words the lifetime of the resource. \refsource{zio:scope} demonstrates usage of these operators, and how scoping affects the order of acquiring and releasing resources.

\begin{algorithm}

\begin{minted}{scala}
def log(msg: String): UIO[Unit] = ZIO.debug(msg)

val intResource: ZIO[Scope, Nothing, Int] = ZIO.acquireRelease(
  acquire = log("acquire int").as(34),
)(release = int => log(s"release $int"))

val stringResource: ZIO[Scope, Nothing, String] = ZIO.acquireRelease(
  acquire = log("acquire string").as("foo"),
)(release = str => log(s"release $str"))

// "acquire int", "acquire string", "release foo", "release 34"
val program1 = ZIO.scoped { intResource *> stringResource }

// "acquire int", "release 34", "acquire string", "release foo"
val program2 = ZIO.scoped(intResource) *> ZIO.scoped(stringResource)
\end{minted}

\caption{Operators for acquiring resources and providing a \inlinecode{Scope} in ZIO. Resources can be scoped to shared, or separate scopes. \label{zio:scope}}
\end{algorithm}

If multiple resources are acquired, they are released in the reverse order. By default releasing resources happens sequentially, but \inlinecode{Scope} also enables to run finalizers in parallel if configured so. 
When using \inlinecode{Scope} with \inlinecode{ZIO.scoped}, finalizers are guaranteed to be executed even when error/defect is encountered, or when the workflow is interrupted.

Traditionally resource management is implemented with \inlinecode{try-finally} statement, where the resource is acquired, before using it in \inlinecode{try} block, and lastly releasing it in the \inlinecode{finally} block. This guarantees that the resource is released, even if error occures after acquiring the resource. Managing resources with \inlinecode{try-finally} lacks in expressivity, composability, and safety compared to higher-level declarative strategy like \inlinecode{Scope}. Firstly, the acquired resource is not visible in the type system, so it is possible to forget to release the resource. Composing acquisition and release of several resources with \inlinecode{try-finally} can be complicated, especially if the acquisition and release must be done in a certain order. When resource is acquired, the lifetime of the resource must be statically determined (by adding a \inlinecode{finally} statement).



\section{Concurrency}
\todo{Siirrä concurrency-osuus background-lukuun?}
Modern applications often use IO a lot, whether it be communicating over the network or interacting with the file system. It is characteristic of IO that a large proportion of time is spent waiting for a response, rather than calculating results utilizing local compute resources, mainly CPU. Today's hardware is capable of executing multiple processed in parallel. In order to increase the throughput and efficiency of an application, many workflows should be executed concurrently to maximize the utilization of the hardware.

Often concurrent workflows must interact with other workflows. A parent workflow might create multiple child workflows to execute some logic in parallel, and combine the results of all workflows, once they are finished. Sometimes a task ought to be split up between concurrent workflows in such way that no two workflows execute the same task. These results of the parallel execution must also be collected together once finished. In some situations, one might \textit{race} workflows i.e. run several workflows in parallel and choose the result that is computed the fastest, discarding all other results. Concurrent workflows sometimes must use a shared resource like mutable data structure, file, or connection to database. Sometimes parallel workflows do not interact with each other in any way, and the focus is to maximize the utilization of the hardware.


\subsubsection{Concurrency problems}
By default the execution order of parallel workflows is nondeterministic, because of how tasks are scheduled (usually) by the operating system to run on actual hardware. Concurrent workflows interacting with each other may be susceptible to race conditions, where the program might behave differently depending on the order of execution. In order to avoid race conditions, explicit countermeasures are required. Logic used to split work between workflows must guarantee that a single task is given only to a single workflow. The logic for collecting the results of parallel workflows must make sure that all results are combined correctly together. The access for a shared resource must be coordinated in a way where only a single or pre-determined number of workflow(s) has access to the resource at given point of time.

At first glance, the problems above may seem trivial, but concurrency really complicates the matter a lot. Very few statements in any programming language are executed with a single CPU instruction, but are usually a combination of several CPU instructions, executed sequentially at different clock cycles. A canonical example of this is the increment operator (\inlinecode{++} or \inlinecode{+=}) found in many languages consists of multiple instructions where the current value of the variable is first read and then set to updated value. If another parallel workflow is updating the same variable at the same time, it might see the value between the two instructions, even though that is rarely the desired behavior. Many bugs related to concurrency are significantly more complex than the mentioned bug.

There are several data structures, constructs, and even CPU instructions for enforcing constraints on concurrent interactions. Atomic compare-and-swap operations enable to observe and update some data, succeeding only if the data was not modified by another workflow in between. Queue is a data structure that could be used to distribute work between multiple workflows. Producers put tasks into the queue and it is guaranteed that each task is only delivered to single a worker. Access to shared resources or parts of the code could be constrained to single workflow at the time with a lock, or mutex, that both grants exclusive access to single workflow at a time. Semaphore is like mutex, but contains a predetermined number of permits instead of just one.

Compare-and-swap operation can fail if the value was updated by another workflow concurrently. In this case the update should be retried, until it succeeds. Usually the retry logic requires significantly more code than a simple variable assignment. When the program has multiple locks that must be held at the same time, the possibility of deadlocks arise. Deadlock is a situation where concurrent workflows are blocked by each other and neither can continue until the other release lock/mutex they are holding.


\subsubsection{Structured concurrency}
\todo{Mihin kohtaan structured concurrency sopii?}
When parent workflow spawns many child workflows to split a task between them, it is common that if one of the children encounter an error, the result cannot be computed at all and thus the results of other sibling workflows are not needed anymore. Similar situation happens when racing workflows, after the first workflow successfully computes a result, the results of other workflows are no longer needed. In both of the situations it would be ideal to cancel the workflows whose results are not needed to preserve compute resources and make sure that no concurrent workflows remain in execution. Traditional concurrency primitives, such as threads, do not offer this kind of control out of the box.

Solution to this is \textit{structured concurrency}, which makes it possible to define clear semantics on if and how a child workflow could outlive its parent. The basic idea of structured concurrency is that there is a way to govern how child workflows are handled when the parent workflow completes (by succeeding or failing), or when there is an error encountered in any of the sibling workflow. For example if a parent workflow that spawns a child workflows and the parent completes or is cancelled before the children are finished, one would like to define whether the children should be awaited, cancelled, or left orphaned.

Native support for structured concurrency in programming languages is still quite rare, but it has been added into programming languages at an accelerating pace. Kotlin added structured concurrency back in 2018~\cite{kotlin-sc}, Swift 5.5 in 2021~\cite{swift-sc} and Java 19 in 2022~\cite{java-sc}. The feature will probably find its way into more programming languages in the future.


\subsubsection{Physical and semantic blocking}
Some function calls might take a long or even indefinite time to complete. This is especially evident when concurrency is involved. There are three main reasons why a function might take a long time to complete; IO, long running CPU intensive computation, or waiting on another workflow to complete. When a function takes a long time to return to it's caller, its said to be \textit{blocking}. Blocking caused by CPU intensive computation is necessary, but blocking, IO bound computations are not utilizing the hardware in a useful way and should be avoided when possible. Blocking can be semantic or physical. Semantic blocking is defines how programs are structured around blocking constructs, and physical blocking determines how the OS and hardware-level resources executing the program are utilized.

Semantic blocking is mandatory since certain operations like, waiting response from a user or remote service can take indefinite time, and the result is required by further computations. Physical blocking in the other hand is not necessary and actually should be avoided, because it prevents the current thread from doing useful work and encourages to create more threads to compensate for the blocked threads (the bad consequences of which are explained below). The line between physical and semantic blocking is often blurry, which is quite understandable considering how they influence the implementation of each other.

There are different techniques for expressing and implementing semantic blocking depending on the programming paradigm, language, and library. A trivial approach is to semantically block on every function call, but often more control over what can be executed concurrently is desired. A common approach is to reverse the responsibility and provide a \textit{continuation} function, which represents the rest of the program, that will be invoked in the future when the result is available. This type of interface is called a callback-based API.

Many languages and libraries have a Future/Promise data type, that is a handle to ongoing computation that should complete in the future, and provides an API to define additional behavior such as timeout, error handling, and interact with the result when the computation completes. Examples of such data types are \inlinecode{Promise} in JavaScript, \inlinecode{Task} in C\# and \inlinecode{Future} in Java.

Another common approach are Monadic APIs, that mostly abstract over blocking behavior in a opaque way. Monadic and Future/Promise based APIs are often implemented in terms of callbacks.

Several programming languages, such as JavaScript, C\# and Kotlin, have specific keywords, usually \textit{async} and \textit{await}, that make it possible to define exactly where the semantic blocking should happen. Compilers and interpreters are able to make a transformation (usually to a state machine or continuation-passing style) from callback-based or Future/Promise API to a one that allows to expose async/await to the programmer.


\subsubsection{Concurrency primitives}
In practice concurrency can be implemented with many different constructs. The lowest-level construct commonly accessible to programming languages is a \textit{thread}. It is a \acronym{OS}{Operating System} level abstraction for concurrent execution. Each thread has its' own stack, instruction pointer and \acronym{CPU}{Central Processing Unit} register values. All threads created by a single process share the same memory space, i.e. are able to read and write shared to same memory blocks.

Threads are run on the actual hardware of the computer, enabling a modern multi-core CPU to execute several workflows in parallel. The OS \textit{schedules} different threads for execution, and after a thread has been executing for a scheduled amount of time, the operating system interrupts the execution and switches the execution to a different thread. The operation where a CPU core switches the execution from one thread to another is called \textit{context switch}. Context switch requires that CPU registers and stack of the previous thread are saved, and respectively registers and stack of the new thread is loaded to the CPU.

% Threads
Traditionally a thread has been the concurrency primitive to turn to when some form of parallelism is required. Threads, however, are not a lightweight construct and they can exist in limited numbers, usually in the thousands. Context switch between threads involves significant amount of work, and thus causes performance overhead. Often context switch defeats many optimizations in contemporary CPUs like caching, pipelining and speculative execution, which in turn amplifies the performance overhead. The situation gets worse when the number of threads increases because the operating system tries to allocate execution time to each thread in a fair way, and the number of context switches increases since each thread has shorter time slot to execute. The issue manifests itself particularly when in highly concurrent scenarios where the computations are IO bound, which is usually the case in web and enterprise applications.

% Thread pools
A common way to constrain the total number of threads and increase their reuse, is to collect multiple threads into a \textit{pool} of threads, where tasks could be submitted for execution instead of operating with individual threads. Once task is submitted to a thread pool, its queued and run once there are available threads. This way many concurrent workflows could be \textit{multiplexed} into smaller number of physical threads. When fewer threads are created and reused by multiple concurrent workflows, the intent is to decrease the number of context switches in the hope of performance gains. There are different types of thread pools depending on how threads in the pool are created. Some thread pools have fixed number of threads, while others grow and shrink dynamically.

% Cooperative scheduling
Thread pools do not solve the problem that when a thread is waiting an IO operation or another thread to complete, the waiting thread is blocked. When a thread blocks, the OS puts it in a waiting state, meaning that it's execution is not continued until the event it is waiting on is triggered. The ideal solution would be that no physical threads are blocked, and blocking is only semantic. This is not possible when the threads are preemptively scheduled by the OS. A solution is to change the scheduling model from preemptive to \textit{cooperative}. Cooperative scheduling means that when a workflow is about to be blocked, it will register itself to be scheduled once all of its dependencies are met, and yield the control to another workflows. In this model no physical threads need to be blocked.

% Event loops
Cooperative scheduling is usually implemented by a runtime environment, programming language, or library, that runs on top of preemptively scheduled OS threads. Event loop is a common pattern to implement cooperative scheduling, and it is used extensively in asynchronous IO or single-threaded environments like JavaScript. The idea is to have a queue that contains computations waiting to get executed, and the event loop picks up and executes tasks from the queue one at a time. Once a task is about the do a blocking operation, it registers a callback. The callback is invoked when the blocking operation completes, and it will add another task to the queue that represents the remaining of the workflow.

% Fibers
Another way to implement cooperative scheduling is \textit{fibers}. They are lightweight threads that are managed and scheduled in the application instead of OS. Each fiber contains a stack and possibly error handlers or thread-local variables, similar to a thread. Fibers require a runtime that schedules fibers to run on actual OS threads. The runtime can multiplex many fibers to run on smaller number of physical threads. Fibers could exist in the hundreds of thousands or millions, and switching execution from one fiber to another is very cheap in comparison to context switch between threads. The fiber runtime can assign a fiber to run on specific CPU core, that will make it easier to reap benefits from CPU optimizations like caches.


\section{ZIO concurrency}
ZIO values are descriptions of a workflow that can be executed in different ways. They can be executed sequentially or concurrently, and the decision can be made after a ZIO workflow is defined. This makes ZIO, or any other IO monad, ideal for high level combinators that allow the programmer to precisely define the concurrency semantics of a computation.

The concurrency model in ZIO is based on fibers. Every operation that waits another ZIO/fiber to complete is semantically blocking and not blocking actual threads. If interacting with code doing blocking IO, ZIO has a separate thread pool dedicated for blocking operations. Extending ZIO's declarative nature is structured concurrency by default and related operators, that enable one to express error handling properties and interruption. Additionally ZIO offers many concurrency primitives such as queues, atomic references and semaphores, as well as software transactional memory, which are not discussed in more depth in this thesis.

Every ZIO workflow is executed by a fiber that is in turn executed by the ZIO runtime that assigns and schedules fibers to be run on actual threads. In ZIO, fiber is a datatype that is a handle to ongoing computation. ZIO program is started on a fiber created by the runtime called \textit{main fiber}. Additional fibers could be created with the \inlinecode{fork} operator on a ZIO workflow. The \inlinecode{fork} operator starts executing the forked fiber concurrently in the background and returns immediately to the original fiber. Other common operations with fibers are to check whether it's finished (\inlinecode{poll}), wait for the result (\inlinecode{join} and \inlinecode{await}), or to interrupt its execution (\inlinecode{interrupt}). Most operations on fibers are effects, and thus return their result inside a ZIO. \refsource{zio:forking} demonstrates forking and joining a fiber.

\begin{algorithm}

\begin{minted}{scala}
val work = ZIO.sleep(1.second) *> ZIO.debug("Work completed")
val parentZIO = for
  childFiber <- work.fork
  _          <- ZIO.debug("Parent forked child fiber")
  _          <- childFiber.join
  _          <- ZIO.debug("Parent joined child fiber")
yield ()

// When executed prints:
// Parent forked child fiber
// Work completed
// Parent joined child fiber
\end{minted}

\caption{Forking and joining a fiber in ZIO. \label{zio:forking}}
\end{algorithm}

Structured concurrency in ZIO is implemented with a fiber \textit{supervision} model. Every forked fiber in ZIO has a scope that determines the maximum lifetime of a fiber. When a scope is closed, all fibers belonging to that scope that have not finished executing are interrupted. The scope is determined at the time of forking, depending on which operator the forking is done with. It is also possible to change the scoping of a fiber after it is forked, but it is somewhat rare. \refsource{zio:fork-operators} introduces different forking operators and their type signatures.

\begin{algorithm}

\begin{minted}{scala}
trait ZIO[-R, +E, +A]:
  def fork: URIO[R, Fiber[E, A]]
  def forkDaemon: URIO[R, Fiber[E, A]]
  def forkScoped: URIO[R & Scope, Fiber[E, A]]
  def forkIn(scope: Scope): URIO[R, Fiber[E, A]]
\end{minted}

\caption{Forking operators on ZIO. \label{zio:fork-operators}}
\end{algorithm}

The default is to scope child fibers to their parent, which is achieved with \inlinecode{fork} operator. In order for a fiber to outlive its parent, a different operator is required. If a fiber should live forever, independently from its parent, \inlinecode{forkDaemon} operator attaches fiber to \textit{global scope} that is closed only when the whole application exits. For finer-grained control over the scope of fiber, its lifetime could be tied to ZIO \inlinecode{Scope}, with \inlinecode{forkScoped} operator, which is scoped to surrounding \inlinecode{Scope} in the ZIO environment, or \inlinecode{forkIn} operator, which takes a \inlinecode{Scope} as an argument. \refsource{zio:fiber-scopes} demonstrates forking fibers in different scopes and their interruption properties.

\begin{algorithm}

\begin{minted}{scala}
def log(msg: String): UIO[Unit] = ZIO.debug(msg)
def hangForever(tag: String): UIO[Nothing] =
  log(s"Start: $tag") *> ZIO.never.onInterrupt(log(s"Stop: $tag"))

val supervision: UIO[Unit] = for
  _     <- hangForever("fork").fork
  _     <- hangForever("forkDaemon").forkDaemon
  scope <- Scope.make
  _     <- hangForever("forkIn").forkIn(scope)
  _     <- ZIO.scoped(hangForever("forkScoped").forkScoped)
  _     <- scope.close(Exit.unit)
yield ()

// Start order(non-deterministic): fork, forkDaemon, forkIn, forkScoped
// Interruption order: forkScoped, forkIn, fork
// forkDaemon is not interrupted
\end{minted}

\caption{Fiber scopes and interruption in ZIO \label{zio:fiber-scopes}}
\end{algorithm}

The fibers of an application can be thought of as a tree where the main fiber is the root node, new child nodes are created by a \inlinecode{fork} operation, and each parent fiber is the root node of its subtree, from which all child fibers branch. When fiber terminates, either by succeeding, failing, or by interruption, all of its descendant fibers are recursively interrupted. After the child fibers have been interruped current fiber's finalizers are executed. Call to interrupt a fiber blocks until the fiber has interrupted all of its children, and all finalizers have finished executing. If a fiber has a large number of descendants with long-running or many finalizers, the interruption could take a significant amount of time. Sometimes it is desired to perform the interruption in the background by a daemon fiber and return immediately to the fiber that initiated the interrupt. This can be achieved by interrupting the fiber with \inlinecode{interruptFork} method or by using \inlinecode{disconnect} combinator on ZIO workflow to make the interruption happen in the background.

Sometimes a fiber is doing critical work, such as disposing acquired resources, that cannot be interrupted without leaving the program in an inconsistent state. These parts of the program should therefore be executed without interruptions. ZIO guarantees that if a fiber, that is executing a section marked as uninterruptible, is interrupted by another fiber, the uninterruptible section is executed to completion despite the interruption. A ZIO workflow can be marked as uninterruptible with \inlinecode{uninterruptible} and \inlinecode{uninterruptibleMask} operators. The former marks whole ZIO workflow as uninterruptible, while the latter gives more control over what parts inside a uninterruptible section are interruptible.

Fibers along with other concurrency primitives are basic building blocks in creating concurrency operators in ZIO. Countless concurrent and parallel combinators can be implemented with forking, joining and interrupting fibers in various ways. Combinators implemented with fibers automatically inherit structured concurrency properties like supervision, scoping and interruption. \refsource{zio:fiber-zippar} demonstrates how \inlinecode{zipPar} concurrency operator can be implemented by using fibers.

\begin{algorithm}

\begin{minted}{scala}
// Actual implementation in ZIO is considerably more complex due to
// environment, errors, race conditions, and other concerns
def zipPar[A, B](left: UIO[A], right: UIO[B]): UIO[(A, B)] =
  for
    fiber1 <- left.fork
    fiber2 <- right.fork
    a      <- fiber1.join
    b      <- fiber2.join
  yield (a, b)

\end{minted}

\caption{\inlinecode{zipPar} implementation with fibers in ZIO. \label{zio:fiber-zippar}}
\end{algorithm}

Fibers are a low-level construct and programming directly with them is error-prone because of possible race conditions. ZIO has numerous built-in high-level concurrency operators (a few of which are presented below) that should be used when possible instead of using fibers. Operators that combine multiple ZIOs in parallel are usually suffixed with \inlinecode{Par} to indicate that the execution happens in parallel. For majority of the operators that combines several independent ZIOs, there is a parallel counterpart that executes in parallel. Listings \ref{zio:binary-combinators} and \ref{zio:multi-combinators} demonstrated ZIO combinators that combine several ZIOs sequentially, whose parallel counterparts include \inlinecode{zipPar}, \inlinecode{foreachPar}, and \inlinecode{collectAllPar}, to name a few. Some operators only make sense to be defined as parallel, such as \inlinecode{race} and its variants, that execute multiple ZIOs and picks the one that succeeds first.

Many combinator operators (like \inlinecode{foreach}, \inlinecode{collectAll}, and every \inlinecode{zip} variant) need the result of each combined ZIO in order to compute a result, and as a result if even one of the ZIOs to be composed fail, the result cannot be computed. In sequential composition this is simple because if a ZIO fails, the execution of subsequent ZIOs won't be started. When composing ZIOs in parallel, this gets a little more complicated. All composed ZIOs start executing in parallel and if any of them fails, the results of others are not needed anymore and they are interrupted. In some situations this interrupting behavior is not desired, and can be avoided by converting ZIOs to infallible, with operators described in Section \ref{zio:error-handling}, before the parallel composition. \refsource{zio:parallel-combinators} demonstrates \inlinecode{zipPar} operator and interruption associated with it.

\begin{algorithm}

\begin{minted}{scala}
// Represents long-running interaction such as network or file system
def work(duration: Duration) = ZIO.sleep(duration)

val fast: IO[String, Int]  = work(50.millis) *> ZIO.fail("oops")
val slow: IO[Nothing, Int] = work(3.seconds) *> ZIO.succeed(34)

// 'slow' is interrupted after 50ms when 'fast' fails
val successInterrupted: IO[String, (Int, Int)] =
  fast.zipPar(slow)

// 'slow' is not interrupted because 'fast' is made infallible
val successNotInterrupted: IO[Nothing, (Either[String, Int], Int)] =
  fast.either.zipPar(slow)
\end{minted}

\caption{Parallel composition of ZIOs with \inlinecode{zipPar} operator. \label{zio:parallel-combinators}}
\end{algorithm}

By default parallel combinators in ZIO have unbounded parallelism, meaning that all composed ZIOs are executed at the same time. Often one would want to limit the amount of parallelism especially with operators, like \inlinecode{foreachPar} or \inlinecode{collectAllPar}, whose parallelism is defined by the size of a collection received as an argument. ZIO has two basic operators for controlling the amount parallelism: \inlinecode{withParallelism} that limits concurrency to a number received as argument, and \inlinecode{withParallelismUnbounded} that removes any limitations to parallelism. These operators only apply to single ZIO workflow, meaning that parallelism is limited only in a specific ZIO. Composing ZIOs with varying parallelism limits preserves the parallelism of each individual ZIO workflow. \refsource{zio:limit-parallelism} demonstrates the usage of operators controlling the amount of parallelism.

\begin{algorithm}

\begin{minted}{scala}
def fetchContent(url: URL): IO[Throwable, String] = ???
val urls: List[URL]                               = ???

val contents: IO[Throwable, List[String]] =
  ZIO.foreachPar(urls)(fetchContent)

// By default all requests are performed in parallel
val unboundedParallelism = contents

// The parallelism is limited to 10 concurrent requests
val boundedParallelism =  unboundedParallelism.withParallelism(10)

// Bounded parallelism can be converted back to unbounded
val unboundedAgain = boundedParallelism.withParallelismUnbounded
\end{minted}

\caption{ZIO operators for controlling the amount of parallelism. \label{zio:limit-parallelism}}
\end{algorithm}


\todo{Kokoa yhteen huomioita ZIO:sta ?}
% Expressivity
% Composability
% Type safety
% Comprehensibility
% Interoperability with existing libraries (even Java)
% Although quite new, few years of production experience from large companies

\chapter{Case description}
\section{Qlik}
\section{Regression testing}

\chapter{Case study}
\section{Design}

\section{Implementation}

\section{Analysis}
\begin{itemize}
    \item Mitä saatiin aikaan? (Analysoi tuloksia ja matkaa mahdollisesti tarkastikin)
    \item Mitä olennaisia asioita kohdattiin?
    \item Hyötyjä / Haittoja?
\end{itemize}
    
\chapter{Conclusion}
Monads as a way to encode effects were discovered in the 90s. It is possible to use monads in a majority of current languages, as long as the language has support for higher-order functions. Assuming a statically typed language, monads also provide an effect system, in addition to modeling effects. Even though monads are not part of mainstream industrial programming, they have been used for a long time and they are quite a mature approach today. Challenges with monads is that they force programs to be written in monadic syntax. Also, combining different monadic effects is not straightforward and necessitates the use of complex programming constructs.

Algebraic effects and handlers are a more recent approach for managing effects that was discovered in the early 2000s; first academic languages appeared in the 2010s. First languages with support for algebraic effects intended for commercial use surfaced in the early 2020s. A productive use of algebraic effects require a language that has native support for them. Such languages usually come with a built-in effect system as well. These languages allow programmers to write effectful programs in direct style, and combining different effects is effortless. Algebraic effects and handlers are, however, a recent practice with many remaining open questions regarding how they should be best included in a programming language.

Programming in a direct style with algebraic effects resembles imperative programming. It can be argued that the direct style of programming is more familiar to the majority of programmers, thus making algebraic effects easier to comprehend than monadic effects. Monadic effects, on the other hand, are far more accessible to the average programmer than algebraic effects, since there are several monadic effect libraries available for different languages. Both approaches enable highly expressive and modular effects; monads with combinators that modify a value representing a computation and algebraic effects with handlers that interpret the effect in a specific manner.

Capability based effects address many shortcomings of monads and algebraic effects. Their research is ongoing, and it is not yet possible to use them in practical applications, since languages with support for them are experimental research languages. Nevertheless, proposals related to effect polymorphism seem to go a long way to make capability based effects more practical and easier to use than other sophisticated approaches for managing side effects.

Compared to unrestricted side effects, monads and algebraic effects provide attractive ways to manage side effects. Regarding the research questions formulated in the introduction, it can be stated that controlling side effects with monads and algebraic effects is clearly more expressive and compositional than unrestricted side effects. This is underlined by how much more convenient it is to implement re-usable logic for effects, such as retries and timeouts, with monads and algebraic effects than it is with unrestricted side effects.

Programs written with monadic or algebraic effects have a tendency to be more declarative than their imperative counterparts with unrestricted side effects. These features facilitate the implementation of modular and resilient programs that are easier to modify and that respond to errors in a clearly defined manner. Concurrency concerns can be alleviated: high-level concurrency makes it easier to implement correct and performant programs when compared to working with traditional imperative low-level primitives, such as threads.

The case study described in this thesis showed that ZIO provides the programmer a good foundation for managing control flows and abstractions, encountered for example in error handling, and leads to declarative and concise programs. Expressing programs in referentially transparent way proved to be beneficial for refactoring, which encourages changing and correcting the program structure as the application evolves. ZIO's benefits are fully realized when approaching problem-solving from a functional programming perspective, which can also pose a weakness: the advantages it provides may not be immediately apparent to programmers who are exclusively familiar with imperative languages.

Algebraic effects with handlers and capability based solutions may eventually turn out to provide better developer ergonomics compared to monads, but currently there is little to none practical experience of using them in commercial software. It remains to be seen whether this sophisticated approach for managing side effects will make their breakthrough in the industry. Eventually it is a trade-off; is a more sophisticated approach perceived useful enough to justify the initial effort of education/learning required. In turn, is it possible to make these more sophisticated approaches more accessible by making them feel more familiar to the practicing programmers, thus requiring less training? In the meantime, ZIO may well be one of the most compelling technologies to try out to get a taste of what these more advanced approaches of handling side effects can offer today.


\printbibliography

\begin{comment}
Important! Create the appendix chapters with command \textbackslash appchapter\{some
name\} instead of \textbackslash chapter\{some name\} for the automagic
page counting to work!
\end{comment}


\appchapter{Liitedokumentti}

Liitteen ohjelmakoodi kuvaa matemaattisen
monadirakenteen pohjalta rakentuvan Haskellin tyyppiluokan. Tyyppiluokan
voi nähdä eräänlaisena abstraktina ohjelmointirajapintana % (API\nomenclature[API]{API}{Application Programming Interface}),
joka muodostaa ohjelmoijalle abstraktin ohjelmointikielen käyttöliittymän
% (UI\nomenclature[UI]{UI}{User Interface}).

\end{document}
