\keywords{functional programming, side effect, algebraic effect, monad, Scala, ZIO}

\begin{abstract}
The management of side effects is a crucial aspect of modern programming, especially in concurrent and distributed systems. This thesis presents different approaches for managing side effects in programming languages, specifically focusing on unrestricted side effects, monads, and algebraic effects and handlers. Unrestricted side effects, used in mainstream imperative programming languages, can make programs difficult to reason about. Monads offer a solution to this problem by describing side effects in a composable and referentially transparent way. Algebraic effects and handlers can address some of the shortcomings of monads by providing a way to model effects in more modular and flexible way. The thesis discusses the advantages and disadvantages of each of these approaches and compares them based on factors such as expressiveness, safety, and constraints they place on how programs must be implemented. The thesis focuses on ZIO, a Scala library for concurrent and asynchronous programming, which revolves around a ZIO monad with three type parameters. With those three parameters ZIO can encode the majority of practically useful effects in a single monad. ZIO takes inspiration from algebraic effects, combining them with monadic effects. The library provides a range of features, such as declarative concurrency, error handling, and resource management. The thesis presents examples of using ZIO to manage side effects in practical scenarios, highlighting its strengths over other approaches. The applicability of ZIO is evaluated by implementing a server side application using ZIO, and analyzing observations from the development process.

\end{abstract}
