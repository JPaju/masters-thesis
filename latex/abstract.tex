\keywords{functional programming, side effect, algebraic effect, monad, Scala, ZIO}

\begin{abstract}
\new{
The management of side effects is a crucial aspect of modern programming, especially in concurrent and distributed systems. This thesis presents different approaches to managing side effects in programming languages, specifically focusing on unrestricted side effects, monads, and algebraic effects and handlers.
Unrestricted side effects, used in mainstream imperative programming languages, can make programs difficult to reason about. Monads offer a solution to this problem by describing side effects in a composable and referentially transparent way. Algebraic effects and handlers are able to address some of the shortcomings of monads by providing a way to model effects in more modular and flexible way.
The thesis focuses on ZIO, a Scala library for concurrent and asynchronous programming, which revolves around a ZIO monad with three type parameters. With those three parameters ZIO is able to encode majority of the practically useful effects in a single monad. ZIO takes inspiration from algebraic effects and combines those with monadic effects. The library provides a range of features such as concurrency primitives, error handling, and resource management.
The thesis discusses the advantages and disadvantages of each approach and compares them based on factors such as expressiveness, safety, and constraints it places. Additionally, the thesis presents examples of using ZIO to manage side effects in practical scenarios, highlighting its strengths over other approaches.
}

\todo{Case study?}

\end{abstract}
