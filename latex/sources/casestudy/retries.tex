\begin{algorithm}

\begin{minted}{scala}
class ApikeyService(keyGen: KeyGen, repo: ApikeyRepository):
  def create(description: String): IO[InvalidDescription, Apikey] =
    type CreateError = DuplicateApikey | InvalidDescription

    val createApikey: IO[CreateError, Apikey] = for
      validDesc   <- validateDesc(description)
      token       <- keyGen.generateKey
      savedApikey <- repo.add(Apikey(validDesc, token))
    yield savedApikey

    // Retry only in the case of DuplicateApikey and at most 10 times
    val policy = Schedule.recurWhile[CreateError] {
      case DuplicateApikey(_)    => true
      case InvalidDescription(_) => false
    } && Schedule.recurs(10)

    // Apply the retry policy to the ZIO that creates the apikey
    val retried: IO[CreateError, Apikey] = createApikey.retry(policy)

    // Change the error type of the retried ZIO:
    // IO[CreateError, Apikey] => IO[InvalidDescription, Apikey]
    // By converting all errors to defects except InvalidDescription
    retried.refineOrDieWith {
      case descError: InvalidDescription => descError
    }(otherErr => RuntimeException(s"Defect in create: $otherErr"))

  def validateDesc(str: String): IO[InvalidDescription, String] = ???
\end{minted}

\caption{Expressing sophisticated retry policies declaratively with \inlinecode{Schedule} and \inlinecode{retry} operator on ZIO. \label{casestudy:retries}}
\end{algorithm}