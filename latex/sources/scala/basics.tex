\begin{algorithm}

\begin{minted}{scala}
trait Foo // Define an interface
class Bar extends Foo // Define a class inheriting from Foo

// Define variables/constants
var mutableFoo: Foo = Bar() // Explicit type is Foo
val immutableBar    = Bar() // Inferred type is Bar
lazy val lazyPlus   = 1 + 1 // Computed lazily and cached

// Type argument here is Int
val genericType: List[Int] = List(1, 2, 3)

// Type parameters are declared between '[' and ']'
def genericMethod[A](a: A): A = a

// Type parameter constraints:
// 'A' must be supertype of 'Bar' and  'B' must be subtype of 'Foo'
def typeBounds[A >: Bar, B <: Foo](a: A): B = ???

// ??? is defined in the standard library. It can replace any
// expression; it's type is Nothing, the bottom type
def `???` : Nothing = throw new NotImplementedError

// => specifies 'by-name' calling convention:
// The parameter n is evaluated every time it is used (2 times here)
def byNameParameter(n: => Int) = n + n
\end{minted}

\caption{Basic syntax of Scala. \label{scala:basics}}
\end{algorithm}