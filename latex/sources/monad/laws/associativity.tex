\begin{algorithm}

    \begin{minted}{scala}
                def pure[A](a: A): Option[A] = Monad[Option].pure(a)
                
                val num1: Option[Int] = Some(1)
                val num2: Option[Int] = Some(2)
                val num3: Option[Int] = Some(3)
                
                val mustBeTrue = sumAll1 == sumAll2
    \end{minted}

    \begin{minipage}{0.40\textwidth}
    \begin{minted}{scala}
def sumAll1: Option[Int] = 
  num1.flatMap(n1 =>
    num2.flatMap(n2 =>
      num3.flatMap(n3 =>
        pure(n1 + n2 + n3))
    )
  )
    \end{minted}
    \end{minipage}
    %
    \hspace{0.05\textwidth}
    %
    \begin{minipage}{0.40\textwidth}
    \vspace{0.05\textwidth}
    \begin{minted}{scala}
    def sumAll2: Option[Int] =
      num1.flatMap(n1 =>
        num2.flatMap(n2 =>
          pure(n1 + n2))
      ).flatMap(sum12 =>
        num3.flatMap(n3 =>
          pure(sum12 + n3))
      )
    \end{minted}
    \end{minipage}

    \caption{Monad associativity law in Scala %
    \label{monad:laws:associativity}}
\end{algorithm}