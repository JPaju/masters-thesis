\chapter{ZIO} \label{zio}
\section{Overview}
\begin{itemize}
    \item Scala-kirjasto sivuvaikutusten hallintaan
    \item Vertical effect rotation
    \item Parempi suorituskyky kuin monad transformereilla (tai ainakin niin väitetään)
    \item Myös muita ominaisuuksia kuten type safe DI, Lokitus, Tracing, Resource handling, Streaming, STM
    \item Ei vaadi merkittävää teoreettista osaamista funktionaalisesta ohjelmoinnista, kuten type classit ja higher kinded types %   - Mainitse että ei tietoisesti käytä type classeja julkisessa API:ssa
\end{itemize}

\section{Error handling}
\begin{itemize}
    \item Typed, lossless, error model even in the presence of async errors
    \item Introduction and elimination of errors
    \item Combination of errors (union)
\end{itemize}

\section{Environment}
\begin{itemize}
    \item Combining two environments (intersection)
    \item Local environment elimination
    \item Effect tracking
\end{itemize}

\section{Concurrency \& Interruption}

\section{Layers}

\section{Managed resources}

\section{STM}

\section{Similar techniques}
\subsection{Tagless Final}
\begin{itemize}
    \item Omaksuttavuus (vaatii ymmärrystä higher kinded tyypeistä ja tyyppiluokista)
    \item Ergonomia
\end{itemize}

\subsection{Monad transformers}
\begin{itemize}
    \item Performance
    \item Omaksuttavuus (vaatii ymmärrystä tyyppiluokista, monadeista ja vaikka mistä)
    \item Ergonomia (type inference)
\end{itemize}

\subsection{Free monad}
\begin{itemize}
    \item Composition
    \item Performance ?
    \item Omaksuttavuus (vaatii ymmärrystä natural transformationeista/tulkin kirjoittamisesta)
\end{itemize}

\begin{algorithm}

\begin{minted}{scala}
val numberZio  = ZIO.succeed(2)
val doubledZio = numberZio.map(n => n * 2)

val sum: UIO[Int] =
  for
    a <- numberZio
    b <- doubledZio
  yield a + b
\end{minted}

\caption{ZIO Basics \label{zio:basics}}
\end{algorithm}
\begin{algorithm}

\begin{minted}{scala}
trait Random:
  def boolean: UIO[Boolean]

val alwaysTrue: ULayer[Random] =
  ZLayer.succeed(new Random { def boolean = UIO.succeed(true) })

val requiresRandom: ZIO[Random, Nothing, String] =
  Random.boolean.map(won => if (won) "Won" else "Lost")

val randomProvided: ZIO[Any, Nothing, String] =
  alwaysTrue(requiresRandom) // "Won"

\end{minted}

\caption{ZIO layers eliminate requirements from computations \label{zio:provide-layer}}
\end{algorithm}